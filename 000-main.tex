%! TEX root = **/000-main.tex
% vim: spell spelllang=en:

%%%%%%%%%%%%%%%%%%%%%%%%%%%%%%%%%%%%%%%%%%%%%%%%%%%%%%%%%%%%%%%%%%%%%%%%%%%%%%%%
% PREAMBLE
%%%%%%%%%%%%%%%%%%%%%%%%%%%%%%%%%%%%%%%%%%%%%%%%%%%%%%%%%%%%%%%%%%%%%%%%%%%%%%%%
\input{001-preamble}

%%%%%%%%%%%%%%%%%%%%%%%%%%%%%%%%%%%%%%%%%%%%%%%%%%%%%%%%%%%%%%%%%%%%%%%%%%%%%%%%
% EXTRA PACKAGES / CONFIG
%%%%%%%%%%%%%%%%%%%%%%%%%%%%%%%%%%%%%%%%%%%%%%%%%%%%%%%%%%%%%%%%%%%%%%%%%%%%%%%%

\usepackage{fancyhdr}

% Set default figure size
\setkeys{Gin}{width=.9\textwidth, keepaspectratio}

% Center figures by default
\makeatletter
\g@addto@macro\@floatboxreset\centering
\makeatother

\usepackage{luacode}
\usepackage{ulem}
\usepackage{dsfont}
\usepackage{tikz}
% \usepackage{tikzexternal}
\usepackage{pgfplots}
\pgfplotsset{compat=1.18}
\pgfmathsetseed{42} % Set seed for random number generation
\usepgfplotslibrary{statistics}

\usetikzlibrary{arrows.meta}
\tikzset{>={Latex[width=3mm,length=3mm]}}

\tikzset{declare function = {
    binom(\k,\n,\p)=\n!/(\k!*(\n-\k)!)*\p^\k*(1-\p)^(\n-\k);
    normal(\x,\mu,\sigma)=\frac{1}{\sqrt{2*\pi*\sigma^2}}*\exp(-\frac{(\x-\mu)^2}{2*\sigma^2});
}}

\newcommand*\GnuplotDefs{
    % set number of samples
    set samples 101;
    %
    % define beta distribution function
    % (copied from <http://gnuplot.sourceforge.net/demo/prob.5.gnu>)
    Binv(p,q)=exp(lgamma(p+q)-lgamma(p)-lgamma(q));
    beta(x,p,q)=p<=0||q<=0?1/0:x<0||x>1?0.0:Binv(p,q)*x**(p-1.0)*(1.0-x)**(q-1.0);
    gammaPdf(x,a,b)=b**a*x**(a-1)*exp(-b*x)/gamma(a);
}

\newenvironment{nscenter}
 {\parskip=0pt\par\nopagebreak\centering}
 {\par\noindent\ignorespacesafterend}


\usetikzlibrary{shapes,arrows,positioning,calc}
\usetikzlibrary{
    pgfplots.dateplot,
}

\usepackage{cancel}
\usepackage{mathrsfs}

\DeclareMathOperator{\argmin}{argmin}
\DeclareMathOperator{\argmax}{argmax}

\usepackage{tcolorbox}
\tcbuselibrary{most}

% \theoremstyle{plain}% from `amsthm'
% \newtheorem{lemma}{Lemma}% from `amsthm'
% \newtheorem{prop}{Proposition}% from `amsthm'

\theoremstyle{definition}% from `amsthm'
\newtheorem{defn}{Definition}% from `amsthm'

\theoremstyle{remark}% from `amsthm'
\newtheorem*{rem}{Remark}% from `amsthm'
\newtheorem{ex}{Example}% from `amsthm'
\newtheorem{prob}{Problem}% from `amsthm'
\newtheorem{ques}{Question}% from `amsthm'
% \newtheorem*{note}{Note}% from `amsthm'
\newtheorem*{hint}{Hint}% from `amsthm'

% \newtcbtheorem[number within=section]{definition}{Definition}%
% {colback=orange!5,colframe=yellow!35!black,fonttitle=\bfseries,parbox=false}{def}

% \newtcbtheorem[number within=section]{theorem}{Theorem}%
% {colback=yellow!5,colframe=lime!35!black,fonttitle=\bfseries,parbox=false}{th}

% \newtcbtheorem[number within=section]{prop}{Proposition}%
% {colback=cyan!5,colframe=cyan!35!black,fonttitle=\bfseries}{prop}

% \newtcbtheorem[number within=section]{recap}{Recap}%
% {colback=red!5,colframe=red!55!black,fonttitle=\bfseries}{rec}

% \newtcbtheorem[number within=section]{exercise}{Exercise}%
% {colback=purple!5,colframe=purple!35!black,fonttitle=\bfseries}{ex}

% \newtcbtheorem[number within=section]{question}{Question}%
% {colback=purple!5,colframe=green!35!black,fonttitle=\bfseries}{qu}

% \newtcbtheorem[number within=section]{example}{Example}%
% {colback=white,colframe=white!35!black,fonttitle=\bfseries,parbox=false}{example}

% \newtcolorbox{marker}[1][]{enhanced,
%   before skip balanced=2mm,after skip balanced=3mm,
%   boxrule=0.4pt,left=5mm,right=2mm,top=1mm,bottom=1mm,
%   colback=yellow!50,
%   colframe=yellow!20!black,
%   sharp corners,rounded corners=southeast,arc is angular,arc=3mm,
%   underlay={%
%     \path[fill=tcbcolback!80!black] ([yshift=3mm]interior.south east)--++(-0.4,-0.1)--++(0.1,-0.2);
%     \path[draw=tcbcolframe,shorten <=-0.05mm,shorten >=-0.05mm] ([yshift=3mm]interior.south east)--++(-0.4,-0.1)--++(0.1,-0.2);
%     \path[fill=yellow!50!black,draw=none] (interior.south west) rectangle node[white]{\Huge\bfseries !} ([xshift=4mm]interior.north west);
%     },
%   drop fuzzy shadow,#1}

\tcbset{%
    theo/.style={%
        enhanced,
        skin=bicolor,
        colbacklower=#1!2,
        sharp corners,
        toprule=0pt, rightrule=0pt, bottomrule=0pt, leftrule=1mm,
        colback=#1!5, colframe=#1!80!black, coltitle=#1!80!black,
        fonttitle=\bfseries,
        detach title, before upper={\tcbtitle\par},
        subtitle style={fontupper=\bfseries,toprule=0pt,bottomrule=0pt,colback={#1!10}},
        parbox=false,
    },
    marker/.style={%
      enhanced,
      parbox=false,
      before skip balanced=2mm,after skip balanced=3mm,
      boxrule=0.4pt,left=5mm,right=2mm,top=1mm,bottom=1mm,
      colback=#1!50,
      colframe=#1!20!black,
      sharp corners,rounded corners=southeast,arc is angular,arc=3mm,
      underlay={%
        \path[fill=tcbcolback!80!black] ([yshift=3mm]interior.south east)--++(-0.4,-0.1)--++(0.1,-0.2);
        \path[draw=tcbcolframe,shorten <=-0.05mm,shorten >=-0.05mm] ([yshift=3mm]interior.south east)--++(-0.4,-0.1)--++(0.1,-0.2);
        \path[fill=#1!50!black,draw=none] (interior.south west) rectangle node[white]{\Huge\bfseries !} ([xshift=4mm]interior.north west);
        },
      % drop fuzzy shadow,
  },
}

\newtcbtheorem[number within=chapter]{definition}{Definition}%
{theo=blue}{def}

\newtcbtheorem[number within=chapter]{theorem}{Theorem}%
{theo=purple}{th}

\newtcbtheorem[number within=chapter]{prop}{Proposition}%
{theo=red}{prop}

\newtcbtheorem[number within=chapter]{lemma}{Lemma}%
{theo=magenta}{lemma}

\newtcbtheorem[number within=chapter]{recap}{Recap}%
{theo=orange}{rec}

\newtcbtheorem[number within=chapter]{exercise}{Exercise}%
{theo=yellow,coltitle=black}{exer}

\newtcbtheorem[number within=chapter]{question}{Question}%
{theo={green!50!black}}{qu}

\newtcbtheorem[number within=chapter]{example}{Example}%
{theo=black,colback=white}{example}

\newtcolorbox{marker}[1][]{marker={yellow!50!white},#1}
\newtcolorbox{important}[1][]{marker=orange,#1}
\newtcolorbox{cnote}[1][]{marker=yellow,#1}
\newtcolorbox{note}[1][]{marker=white,#1}
%%%%%%%%%%%%%%%%%%%%%%%%%%%%%%%%%%%%%%%%%%%%%%%%%%%%%%%%%%%%%%%%%%%%%%%%%%%%%%%%
% METADATA
%%%%%%%%%%%%%%%%%%%%%%%%%%%%%%%%%%%%%%%%%%%%%%%%%%%%%%%%%%%%%%%%%%%%%%%%%%%%%%%%

% remove when using \maketitle:
\renewcommand\and{\\[\baselineskip]}

\title{Advanced Statistical Modeling}
\author{Aleix Boné}
\date{Fall 2022}

\makeindex

\begin{document}
\newcommand{\iemph}[1]{\index{#1}\emph{#1}}
%%%%%%%%%%%%%%%%%%%%%%%%%%%%%%%%%%%%%%%%%%%%%%%%%%%%%%%%%%%%%%%%%%%%%%%%%%%%%%%%
% TITLE
%%%%%%%%%%%%%%%%%%%%%%%%%%%%%%%%%%%%%%%%%%%%%%%%%%%%%%%%%%%%%%%%%%%%%%%%%%%%%%%%

% Default title or use titlepage.tex

%\maketitle
\pagestyle{empty}

\makeatletter
\begin{tikzpicture}[
		remember picture,
		overlay,
		important line/.style={thick,ForestGreen!15,thick},
		dashed line/.style={dashed,ForestGreen!15,thick},
		leftNode/.style={circle,minimum width=.5ex, fill=ForestGreen!15,draw},
		rightNode/.style={rectangle,minimum width=.5ex, fill=ForestGreen!15,thick,draw},
	]
	%%%%%%%%%%%%%%%%%%%% Background %%%%%%%%%%%%%%%%%%%%%%%%
    \begin{scope}[blend mode=hard light]
	\fill[ForestGreen] (current page.south west) rectangle (current page.north east);

    \node[anchor=north,inner sep=0pt] at ($(current page.north)-(0,1)$) {
        \includegraphics[width=0.9\textwidth,
        ]{logo-white}
    };
    \end{scope}

	\pgfmathsetseed{1234}
	\begin{axis}[
			at={(current page.south west)},
			width=\paperwidth,
			height=\paperheight,
            xmin=-100,xmax=50,
            clip=false,
            ymin=-50,ymax=200,
            ticks=none,
            axis lines=none,
		]
		\pgfmathsetmacro{\sep}{20};
		\pgfmathsetmacro{\xorig}{120};
		\pgfmathsetmacro{\rot}{-60};

		\begin{scope}[rotate around={\rot:(0,0)}]
			\draw[dashed line] (-300, -\sep) -- (300, -\sep);
			\draw[dashed line] (-300, \sep) -- (300, \sep);
			\draw[important line] (-300, 0) -- (300, 0);

            \node[leftNode,label={[ForestGreen!3]:$x_1$},name=x1] at (-20, \sep) {};
			\node[rightNode,label={[ForestGreen!3]$x_2$},name=x2] at (-22, -\sep) {};

			\node[leftNode,label={[ForestGreen!3]$x_3$},name=x3] at (-10, -\sep*3/2) {};
			\node[leftNode,label={[ForestGreen!3]$x_5$},name=x5] at (-27, -\sep/3) {};

			\node[rightNode,label={[ForestGreen!3]$x_4$},name=x4] at (10, -\sep*2/3) {};

			\coordinate (origin) at (0, 0);
			\coordinate (left) at (0, \sep);
			\coordinate (right) at (0, -\sep);

			\coordinate (x3l) at (x3 |- left);
			\coordinate (x4r) at (x4 |- right);
			\coordinate (x5l) at (x5 |- left);

			\begin{scope}[color=ForestGreen!10]
				\draw (x3) -- (x3l) {};
				\draw (x4) -- (x4r) {};
				\draw (x5) -- (x5l) {};
				\node[anchor=north] at ($(x3)!0.5!(x3l)$) {$\xi_3$};
				\node[anchor=north] at ($(x4)!0.5!(x4r)$) {$\xi_4$};
				\node[anchor=north] at ($(x5)!0.5!(x5l)$) {$\xi_5$};
			\end{scope}

			\draw[<->,ForestGreen!5,line width=2pt] (-44,\sep) -- (-44,-\sep);
			\node[ForestGreen!9,anchor=south,rotate=\rot+90] at (-44.25,0) {margin\;\; = $\frac{2}{\lVert \omega \rVert}$};

			\pgfplotsinvokeforeach{0.00,0.1,...,1.00}{
				\node [leftNode] at (rand*40-20,\sep+rnd*40) {};
				\node [rightNode] at (rand*40,-\sep-rnd*40) {};
			}

            \coordinate (p1) at (-70, \sep);
            \coordinate (p0) at (-70, 0);
            \coordinate (p-1) at (-70, -\sep);

		\end{scope}
		\node[ForestGreen!5,anchor=south east,rotate=\rot] at (p1) {$\omega^T x + b = 1$};
		\node[ForestGreen!5,anchor=south east,rotate=\rot] at (p0) {$\pi:\,\omega^T x + b = 0$};
		\node[ForestGreen!5,anchor=south east,rotate=\rot] at (p-1) {$\omega^T x + b = -1$};
	\end{axis}

	% \foreach \i in {2.5,...,22}
	% {
	%     \node[rounded corners,ForestGreen!60,draw,regular polygon,regular polygon sides=7, minimum size=\i cm,ultra thick] at ($(current page.west)+(2.5,-5)$) {} ;
	% }

	% %%%%%%%%%%%%%%%%%%%% Background Polygon %%%%%%%%%%%%%%%%%%%%
	% \foreach \i in {0.5,...,22}
	% {
	% \node[rounded corners,ForestGreen!60,draw,regular polygon,regular polygon sides=7, minimum size=\i cm,ultra thick] at ($(current page.north west)+(2.5,0)$) {} ;
	% }

	% \foreach \i in {0.5,...,22}
	% {
	% \node[rounded corners,ForestGreen!90,draw,regular polygon,regular polygon sides=7, minimum size=\i cm,ultra thick] at ($(current page.north east)+(0,-9.5)$) {} ;
	% }


	% \foreach \i in {21,...,6}
	% {
	% \node[ForestGreen!85,rounded corners,draw,regular polygon,regular polygon sides=7, minimum size=\i cm,ultra thick] at ($(current page.south east)+(-0.2,-0.45)$) {} ;
	% }


	%%%%%%%%%%%%%%%%%%%% Title of the Report %%%%%%%%%%%%%%%%%%%%
	\node[left,ForestGreen!5,minimum width=0.725*\paperwidth,minimum height=3cm, rounded corners] at ($(current page.north east)+(0,-6.0)$)
	{
		{\fontsize{25}{30} \selectfont \bfseries Advanced}
	};
	\node[left,ForestGreen!5,minimum width=0.725*\paperwidth,minimum height=3cm, rounded corners] at ($(current page.north east)+(0,-7.5)$)
	{
		{\fontsize{25}{30} \selectfont \bfseries Statistical Modeling}
	};

	%%%%%%%%%%%%%%%%%%%% Subtitle %%%%%%%%%%%%%%%%%%%%
	\node[left,ForestGreen!10,minimum width=0.725*\paperwidth,minimum height=2cm, rounded corners] at ($(current page.north east)+(0,-9)$)
	{
		{\huge \textit{Lecture Notes}}
	};

	%%%%%%%%%%%%%%%%%%%% Author Name %%%%%%%%%%%%%%%%%%%%
	\node[left,ForestGreen!5,minimum width=0.725*\paperwidth,minimum height=2cm, rounded corners] at ($(current page.north east)+(0,-11)$)
	{
		{\Large \textsc{\@author}}
	};

	%%%%%%%%%%%%%%%%%%%% Year %%%%%%%%%%%%%%%%%%%%
\node[rounded corners,fill=ForestGreen!70,text =RoyalBlue!5,regular polygon,regular polygon sides=6, minimum size=2.5 cm,inner sep=0,ultra thick,align=center] at ($(current page.west)+(2.5,-5)$) {\LARGE \bfseries 2022};

\end{tikzpicture}
\makeatother

\include{005-titlepage}

%%%%%%%%%%%%%%%%%%%%%%%%%%%%%%%%%%%%%%%%%%%%%%%%%%%%%%%%%%%%%%%%%%%%%%%%%%%%%%%%
% TOC & lists
%%%%%%%%%%%%%%%%%%%%%%%%%%%%%%%%%%%%%%%%%%%%%%%%%%%%%%%%%%%%%%%%%%%%%%%%%%%%%%%%

\pagenumbering{Roman}

%\setcounter{tocdepth}{2}
\tableofcontents \vfill
\listoffigures \listoftables \clearpage

\pagenumbering{arabic}

%%%%%%%%%%%%%%%%%%%%%%%%%%%%%%%%%%%%%%%%%%%%%%%%%%%%%%%%%%%%%%%%%%%%%%%%%%%%%%%%
% SECTIONS
%%%%%%%%%%%%%%%%%%%%%%%%%%%%%%%%%%%%%%%%%%%%%%%%%%%%%%%%%%%%%%%%%%%%%%%%%%%%%%%%

% Paragraph spacing (placed after ToC)
\setlength{\parskip}{1em plus 0.5em minus 0.2em}
%\setlength{\parindent}{0pt}

% \selectcolormodel{gray}

\setlength{\headheight}{14.5pt}
\pagestyle{fancy}

% Bayes

\section{Bayesian Model}

\begin{theorem}{Bayes Theorem}{bayes}
	\begin{align}
		P(A \mid B) & = \frac{P(B \mid A)P(A)}{P(B)}                             \\
		P(A \mid B) & = \frac{P(B \mid A)P(A)}{\sum_{i=1}^n P(B \mid A_i)P(A_i)}
	\end{align}
\end{theorem}

\begin{definition}{Statistical Model}{}
	A \iemph{Statistical Model} $M$ is a list of Probability models $P$ indexed by
	a parameter $\theta$ (which can be a scalar, vector, matrix, etc.) that
	we know belongs to a \iemph{parameter space} $\Omega$.
	\\[1em]
	Formally:
	\begin{align}
		M = \{ P(\tilde{y}\mid\theta), \; \theta \in \Omega \}
	\end{align}
\end{definition}

Given a Statistical Model and a Data set $\tilde{y}$, we want to find
which probability model $P(\tilde{y}\mid\theta^*)$ is the most likely to
have generated the data $\tilde{y}$. Where $\theta^* \in \Omega$ is the true
value of the parameter.

\subsection{The four problems of Statistics}
\subsubsection{Data Collection}
Design how to collect the data.

This leads to choose the statistical model from which you will observe the data.
The statistical model will be determined by the nature of the data
(discrete or continuous) and by the type of sampling.

\subsubsection{Model Validation}

After choosing the model and collecting the data, we need to decide if
the model is valid or not.

That is, we have to decide if the probability model that generated the data
belongs to $M$ and therefore the model is correct.

\subsubsection{Statistical Inference}

After collecting the data and checking the model, Statistical Inference
tries to guess the true value of the parameter $\theta^*$. That is, it tries
to guess the probability model that generated the data.

It mainly consists of:
\begin{enumerate}
	\item Point estimation: Guessing the value of the parameter $\theta^*$.
	\item Interval estimation: Guessing the interval of values of the parameter $\theta^*$.
	\item Hypothesis testing: Guessing if the parameter $\theta^*$ belongs to a given set.
	\item Prediction: Guessing the value of a new data point.
\end{enumerate}

\subsubsection{Results Presentation}

Finally, the results must be interpreted and displayed in an understandable way
according to the target audience.

\subsection{Bayesian Model}

\begin{definition}{Bayesian model}{}{}
	A \iemph{Bayesian Model} is a Statistical Model where $\theta$ is treated as a random variable
	where the probability of $\theta$ is given by a prior distribution $\pi(\theta)$,
	before looking at the data (it captures prior knowledge about $\theta$).

	We call $\pi(\theta)$ the \iemph{prior distribution}.

	A \emph{Bayesian Model} is a list of probability distributions sorted from most
	likely to least likely according to the prior distribution $\pi(\theta)$:
	\begin{equation}
		M_B = \{ p(\tilde{y}\mid\theta), \; \theta \sim \pi(\theta), \; \theta \in \Omega \}
	\end{equation}
\end{definition}

\begin{definition}{Likelihood function}{}{}
	All the information that the data $y$ has about the parameter $\theta$ is in the
	\iemph{likelihood function} $L_y(\theta)$.

	The likelihood function is a function proportional to the probability distribution
	of $\tilde{y},\; p(\tilde{y}\mid\theta)$, evaluated at the data $p(y \mid \theta)$
	and expressed as a function of $\theta$:
	\begin{equation}
		L_y(\theta) \propto p(y \mid \theta)
	\end{equation}
\end{definition}

The likelihood function sorts the parameter space $\Omega$ from most likely to least,
so if $L_y(\theta_1) > L_y(\theta_2)$ it means that according to the data $y$,
the parameter $\theta_1$ is more likely than $\theta_2$.

\begin{example}{Probability of head after 10 coin tosses}{}
	\begin{alignat*}{1}
		\text{Statistical model:}   & \; \tilde y \mid \theta \sim \text{Binomial}(10, \theta), \; \theta \in [0,\,1] \\
		\text{where:}               & \; \theta \text{ is the probability of head}                                    \\
		\span \text{after 10 coin tosses, we observed 6 heads:} \; y = 6                                              \\
		\text{Likelihood function:} & \; L_y(\theta) \propto p(y \mid \theta) = \binom{10}{6} \theta^6 (1-\theta)^4
		\propto \theta^6 (1-\theta)^4
	\end{alignat*}
	%
	\begin{nscenter}
		\begin{tikzpicture}
			\begin{axis}[
					samples at={0,0.01,...,1},
					width=0.8\textwidth,
					height=0.3\textwidth,
					ytick=\empty,
					xmin=0,
					xmax=1,
					xlabel=$\theta$,
					legend pos=north west,
				]
				\addplot[mark=none,line width=2pt] {binom(6,10,x)}; \addlegendentry{$L_y(\theta)$}
				\draw[red,dashed,line width=2pt] (0.6,1) -- (0.6,0);
				\addlegendimage{draw=red,dashed,line width=2pt}; \addlegendentry{$\theta^* = 0.6$}
			\end{axis}
		\end{tikzpicture}
	\end{nscenter}
	The maximum of the likelihood function is at $\theta = 0.6$. So according to the data,
	$\theta^*$ is more likely to be 0.6 than any other value.
\end{example}

\begin{definition}{Standarized Likelihood function}{}{}
	It's the likelihood function divided by its integral over the parameter space $\Omega$,
	so that it integrates to 1 and it's a probability distribution:
	\begin{equation}
		\tilde{L}_y^{std}(\theta) = \frac{L_y(\theta)}{\int_{\Omega} L_y(\theta) d\theta}
	\end{equation}
\end{definition}


\begin{recap}{}{}{}
	A Statistical model $M$ is a (not sorted) list of probability models, one for
	each value of the parameter $\theta \in \Omega$:
	\begin{equation*}
		M = \{ p(\tilde{y}\mid\theta), \; \theta \in \Omega \}
	\end{equation*}

	\begin{itemize}
		\item The \emph{prior distribution} $\pi(\theta)$ sorts the probability models of $M$
		      using \emph{prior information} about the parameter $\theta$.
		\item The \emph{likelihood function} $L_y(\theta)$ sorts the probability models of $M$
		      using the \emph{data information} $y$.
	\end{itemize}
\end{recap}

\subsection{Posterior Distribution}

\begin{definition}{Posterior Distribution}{}{}
	The posterior distribution $\pi(\theta \mid y)$ shows the information we have
	about the parameter $\theta$ after looking at the data $y$.

	It is calculated using Bayes Theorem \ref{th:bayes}:
	\begin{equation}
		\pi(\theta \mid y) = \frac{p(y,\,\theta)}{p(y)} = \frac{p(y \mid \theta) \pi(\theta)}{p(y)}
	\end{equation}

	Where $p(y)$ is the \iemph{marginal distribution} of $\tilde{y}$ evaluated at the
	data $y$, which we will also call the \iemph{prior predictive distribution} evaluated at
	the data:
	\begin{equation}
		p(\tilde y = y) = p(y) = \int_{\Omega} p(y,\,\theta) d\theta = \int_{\Omega} p(y \mid \theta) \pi(\theta) d\theta
	\end{equation}

	\begin{note}
		$p(y)$ is a constant, also called the \iemph{normalization constant}, which
		makes the posterior distribution integrate to 1.
	\end{note}

	From this it follows that the posterior distribution is proportional to the
	\emph{product} of the \emph{prior distribution} $\pi(\theta)$ and the
	\emph{likelihood function} $L_y(\theta)$:
	\begin{equation}
		\pi(\theta \mid y) \propto \pi(\theta) L_y(\theta)
	\end{equation}
\end{definition}

\begin{note}
	If $\pi(\theta)$ is a uniform distribution, the posterior distribution is the
	standardized likelihood function $L_y^{std}(\theta)$.
\end{note}

\begin{example}{Probability of head after 10 coin tosses $y=7$}{}
	We assume that the probability of head is close to 50\%, so we chose an
	informative prior distribution:
	\begin{alignat*}{1}
		\text{Statistical model:}      & \; \tilde y \mid \theta \sim \text{Binomial}(10, \theta), \; \theta \in [0,\,1] \\
		\text{Prior distribution:}     & \; \theta \sim \pi(\theta) = \text{Beta}(50,\,50)                               \\
		% \text{where:}             & \; \theta \text{ is the probability of head}                                    \\
		\span \text{after 10 coin tosses, we observed 7 heads:} \; y = 7                                                 \\
		% \span \text{From $y$ we calculate likelihood and posterior:} \\
		\text{Likelihood function:}    & \; L_y(\theta) \propto p(y \mid \theta) = \binom{10}{7} \theta^7 (1-\theta)^3
		\propto \theta^7 (1-\theta)^3                                                                                    \\
		\text{Posterior distribution:} & \; \pi(\theta \mid y) = \text{Beta}(57,\,53)
	\end{alignat*}
	%
	\begin{nscenter}
		\begin{tikzpicture}
			\begin{axis}[
					samples at={0,0.01,...,1},
					width=0.8\textwidth,
					height=0.33\textwidth,
					ytick=\empty,
					xmin=0,
					xmax=1,
					xlabel=$\theta$,
					legend pos=north west,
				]
				\addplot [mark=none,dashed,blue,line width=2pt]gnuplot [raw gnuplot] {
				\GnuplotDefs
				plot [x=0:1] beta(x,50,50);
				}; \addlegendentry{$\pi(\theta)$}
				\addplot[mark=none,blue,line width=2pt] gnuplot [raw gnuplot] {
						\GnuplotDefs
						plot [x=0:1] beta(x,57,53);
					}; \addlegendentry{$\pi(\theta \mid y)$}
				\addplot[mark=none,dotted,red,line width=2pt] {binom(7,10,x)*10}; \addlegendentry{$L_y(\theta)$}
			\end{axis}
		\end{tikzpicture}
	\end{nscenter}
\end{example}

\begin{example}{Probability of head after 10 coin tosses $y=7$ v2}{}
	We take a less-informative prior distribution, since we are not sure that the coin
	is fair:
	\begin{alignat*}{1}
		\text{Statistical model:}      & \; \tilde y \mid \theta \sim \text{Binomial}(10, \theta), \; \theta \in [0,\,1] \\
		\text{Prior distribution:}     & \; \theta \sim \pi(\theta) = \text{Beta}(2,\,2)                                 \\
		% \text{where:}             & \; \theta \text{ is the probability of head}                                    \\
		\span \text{after 10 coin tosses, we observed 7 heads:} \; y = 7                                                 \\
		% \span \text{From $y$ we calculate likelihood and posterior:} \\
		\text{Likelihood function:}    & \; L_y(\theta) \propto p(y \mid \theta) = \binom{10}{7} \theta^7 (1-\theta)^3
		\propto \theta^7 (1-\theta)^3                                                                                    \\
		\text{Posterior distribution:} & \; \pi(\theta \mid y) = \text{Beta}(9,\,5)
	\end{alignat*}
	\vspace{-1em}
	\begin{nscenter}
		\begin{tikzpicture}
			\begin{axis}[
					samples at={0,0.01,...,1},
					width=0.8\textwidth,
					height=0.33\textwidth,
					ytick=\empty,
					xmin=0,
					xmax=1,
					xlabel=$\theta$,
					legend pos=north west,
				]
				\addplot [mark=none,dashed,blue,line width=2pt]gnuplot [raw gnuplot] {
				\GnuplotDefs
				plot [x=0:1] beta(x,2,2);
				}; \addlegendentry{$\pi(\theta)$}
				\addplot[mark=none,blue,line width=2pt] gnuplot [raw gnuplot] {
						\GnuplotDefs
						plot [x=0:1] beta(x,9,5);
					}; \addlegendentry{$\pi(\theta \mid y)$}
				\addplot[mark=none,dotted,red,line width=2pt] {binom(7,10,x)*10}; \addlegendentry{$L_y(\theta)$}
			\end{axis}
		\end{tikzpicture}
	\end{nscenter}
	\begin{note}
		With a less informative prior, the posterior distribution is more affected by the
		likelihood function. If we were to use a flat prior, the posterior would be equal
		to the likelihood function.
	\end{note}
\end{example}

\begin{recap}{}{}
	\centering
	\begin{tikzpicture}[every node/.style={rectangle,draw,fill=white}]
		\node[align=center] (model) {%
			Bayesian model\\[1em]
			$\begin{aligned}
					\tilde y \mid \theta & \sim p(\tilde y \mid \theta) \\
					\theta               & \sim \pi(\theta)
				\end{aligned}$};

		\node[below=of model] (bayes) {Bayes' theorem};
		\node (data) [left=of bayes] {Data $y$};

		\node[below=of bayes,align=center] (updated) {%
			Bayesian model (updated)\\[1em]
			$\begin{aligned}
					\tilde y \mid \theta & \sim p(\tilde y \mid \theta) \\
					\theta \mid y        & \sim \pi(\theta \mid y)
				\end{aligned}$};

		\node[right=of updated,align=center,draw=none,fill=none] (post) {Posterior distribution \\[1em]
			$ \pi(\theta \mid y) = \frac{L_y(\theta) \pi(\theta)}{\int L_y(\theta)\pi(\theta)d\theta} $};

		\draw[->,line width=5pt] (model) -- (bayes) -- (updated);
		\draw[->,line width=5pt] (data) -- (bayes);
		\draw[->,line width=3pt,dashed] (updated.345) -- (post);
	\end{tikzpicture}
\end{recap}

\subsection{Prior and posterior predictive distributions}%

\begin{definition}{Prior predictive distribution}{}
	The \iemph{marginal distribution} of $\tilde{y}$, called the \iemph{prior predictive distribution},
	represents our knowledge and uncertainty about the sample space $\tilde{y}$ before looking at the data $y$:
	\begin{equation}
		p(\tilde y) = \int_{\Omega} p(\tilde y ,\, \theta) d\theta = \int_{\Omega} p(\tilde y \mid \theta) \pi(\theta) d\theta
	\end{equation}

	\paragraph{Observations}
	\begin{itemize}
		\item $p(\tilde y) = E_{\pi(\theta)}[p(\tilde y \mid \theta)]$ is a weighted average of the
		      probability models of $M$ where the weights are determined by the prior distribution $\pi(\theta)$.
		\item $p(\tilde y)$ is a probability distribution that allows us to make predictions before
		      observing the data.
	\end{itemize}
\end{definition}

If we know something about the parameters, it means that we have some prior information about the data
and vice versa. The prior predictive distribution translates the information about the
parameter space into the sample space.

The prior's parameters can be chosen in two ways:
\begin{itemize}
	\item Prior distribution $\pi(\theta)$ if the information we have is about the parameter space.
	\item Prior predictive distribution $p(\tilde y)$ if the information we have is about the sample space.
\end{itemize}

\begin{definition}{Posterior predictive distribution}{post-pred}
	Once the data $y$ is observed, we update the model to reflect the new information,
	changing the prior distribution $\pi(\theta)$ into the posterior distribution $\pi(\theta \mid y)$.
	Which means that the prior predictive distribution $p(\tilde y)$ is replaced by the
	\iemph{posterior predictive distribution} $p(\tilde y \mid y)$:
	\begin{equation}
		p(\tilde y \mid y) = \int_{\Omega} p(\tilde y \mid \theta) \pi(\theta \mid y) d\theta
	\end{equation}

	\paragraph{Observations}
	\begin{itemize}
		\item $\pi(\theta y \mid y)$ shows everything we know about the parameter.
		\item $p(\tilde y \mid y)$ shows everything we know about the data behavior.
	\end{itemize}
\end{definition}

Predictive distributions can be calculated in two ways:
\begin{itemize}
	\item Analytically: solving integrals:
	      \begin{align*}
		      p(\tilde y)        & = \int_{\Omega} p(\tilde y \mid \theta) \pi(\theta) d\theta        \\
		      p(\tilde y \mid y) & = \int_{\Omega} p(\tilde y \mid \theta) \pi(\theta \mid y) d\theta
	      \end{align*}
	\item Numerically (by simulation): sampling from the posterior distribution:
	      \begin{enumerate}
		      \item Decide on a number of simulations $M$ (the larger the better).
		      \item Sample $\theta^{(j)}$ from $\pi(\theta \mid y)$.
		      \item Sample $\tilde y^{(j)}$ from $p(\tilde y \mid \theta^{(j)})$.
		      \item Repeat $j=1,\ldots,M$.
	      \end{enumerate}
	      These values $\tilde y^{(j)}$ can be used to calculate everything we want to know:
	      mean, variance, quantiles, shape of the distribution, etc.
\end{itemize}

\begin{recap}{}{}
	\centering
	\begin{tikzpicture}[every node/.style={rectangle,draw,fill=white}]
		\node[align=center] (model) {%
			Bayesian model\\[1em]
			$\begin{aligned}
					\tilde y \mid \theta & \sim p(\tilde y \mid \theta) \\
					\theta               & \sim \pi(\theta)
				\end{aligned}$};

		\node[below=of model] (bayes) {Bayes' theorem};
		\node (data) [left=of bayes] {Data $y$};

		\node[below=of bayes,align=center] (updated) {%
			Bayesian model (updated)\\[1em]
			$\begin{aligned}
					\tilde y \mid \theta & \sim p(\tilde y \mid \theta) \\
					\theta \mid y        & \sim \pi(\theta \mid y)
				\end{aligned}$};

		\node[right=of model,draw=none,fill=none] (prior) {
			$p(\tilde y)$ (prior predictive dist.)
		};
		\node[right=of updated,draw=none,fill=none] (post) {
			$p(\tilde y \mid y)$ (posterior predictive dist.)
		};

		\draw[->,line width=5pt] (model) -- (bayes) -- (updated);
		\draw[->,line width=5pt] (data) -- (bayes);
		\draw[->,line width=3pt,dashed] (model) -- (prior);
		\draw[->,line width=3pt,dashed] (updated) -- (post);
	\end{tikzpicture}
\end{recap}

\subsection{Choosing the prior}

The prior distribution $\pi(\theta)$ is a subjective choice. It should be based on previous
studies and subjective knowledge of experts. In case of not having information or
not being able to agree on the knowledge of experts, we will use a \iemph{non-informative prior}.

\begin{definition}{Conjugate prior}{}
	A prior distribution $\pi(\theta)$ is the \iemph{conjugate} prior of a statistical model
	if the posterior distribution $\pi(\theta \mid y)$ is of the same family as the prior.

	\begin{table}[H]
		\caption{Examples of conjugate priors}
		\colorbox{white}{%
			\begin{tabular}{ccc}
				\toprule
				Model                & Prior                 & Posterior                        \\
				\midrule
				Binomial($n,\theta$) & Beta($\alpha,\beta$)  & Beta($\alpha + y,\beta + n - y$) \\
				Poisson($\lambda$)   & Gamma($\alpha,\beta$) & Gamma($\alpha + y,\beta + 1$)    \\
				\bottomrule
			\end{tabular}%
		}
	\end{table}

\end{definition}

\paragraph{Informative priors}
To chose informative priors, we can either do trial and error, drawing
$\pi(\theta)$ or solving a system of equations based on moments and or quantiles.

\paragraph{Non-informative priors}
The most common are a flat prior (Laplace's prior) and a conjugate prior with
\emph{huge} variance (e.g. Gamma($\alpha=0.001,\beta=0.001$)).

%! TEX root = **/000-main.tex

\section{Bayesian Inference}

\subsection{Posterior distribution as an estimator}

The posterior distribution is a compromise between the prior
distribution (the information before observing the data) and the
likelihood function (the data information).

The posterior distribution $\pi(\theta \mid y)$ is the natural
Bayesian estimator for $\theta$ given the data $y$.

\subsection{Point Estimation}
Any measure of the location of $\pi(\theta \mid y)$ will serve as a point
estimate:
\begin{align*}
    \hat{\theta}_{pe} &= E(\theta \mid y) = \int \theta \pi(\theta \mid y) \partial\theta \\
    \hat{\theta}_{pme} &\text{ is such that } \int_{-\infty}^{\hat{\theta}_{pme}} \pi(\theta \mid y) \partial\theta = \frac{1}{2} \\
    \hat{\theta}_{pmo} &= \argmax_{\theta} \pi(\theta \mid y) \\
\end{align*}

This can also be computed numerically by simulation. Let $\theta^{(1)}, \dots , \theta^{(M)}$ be
simulations of $\theta$ from $\pi(\theta \mid y)$, then:
\begin{equation}
    \hat{\theta}_{pe} = \frac{1}{M} \sum_{i=1}^M \theta^{(i)}
\end{equation}

\paragraph{Observation} We can also define a prior point estimator using the prior distribution
instead of the posterior distribution.

\subsection{Interval Estimation}
A posterior credibility (or probability) interval of $p$ for $\theta$, $IC_p$ is
any region of $\Omega$ such that:
\begin{equation}
    p(\theta \in IC_p \mid y) = \int_{IC_p} \pi(\theta \mid y) \partial\theta = p
\end{equation}
Usually we use intervals based on percentiles.

\begin{example}{Confidence interval}{}
	\begin{nscenter}
		\begin{tikzpicture}
			\begin{axis}[
					samples at={0,0.01,...,1},
					width=0.8\textwidth,
					height=0.33\textwidth,
					ytick=\empty,
					xmin=0,
					xmax=5,
					xlabel=$\theta$,
					legend pos=north west,
				]
				\addplot [mark=none,black] gnuplot [raw gnuplot] {
				\GnuplotDefs
				plot [x=0:5] beta(x/5,9,5);
				}; \addlegendentry{$\pi(\theta \mid y)$}
				\addplot [mark=none,fill=blue,opacity=0.3,forget plot]gnuplot [raw gnuplot] {
				\GnuplotDefs
				plot [x=2:4.3] beta(x/5,9,5);
				} \closedcycle;
            \addlegendimage{area legend,fill=blue,opacity=0.3}; \addlegendentry{$p$}
            \path (2,0) coordinate (L) (4.3,0) coordinate (R);
			\end{axis}
            \draw[decorate,decoration={brace,raise=1.9em}] (R) -- (L) node[midway,below=2.2em] {$CI_p$};
		\end{tikzpicture}
	\end{nscenter}
    \begin{note}
        For the CI to be valid, the two tails must have equal area $(1-p)/2$.
    \end{note}
\end{example}

\subsection{Prediction}

We use the posterior predictive distribution to make predictions about future observations.
See definition~\ref{def:post-pred}.

\subsection{Hypothesis Testing}

Given a Bayesian model: $M = \{p(\tilde \theta \mid \theta),\,\theta \in \Omega\},\; \pi(\theta)$,
we split the parameter space into two disjoint regions: $\Omega \equiv \Omega_0 \cup \Omega_1$
and we want to decide to which subspace $\Omega_0$ or $\Omega_1$ the parameter $\theta$ belongs.
We formulate the hypothesis as:
\begin{align*}
    H_0:\; \theta \in \Omega_0 \\
    H_1:\; \theta \in \Omega_1
\end{align*}

After observing the data $y$, we can compute the posterior distribution for each hypothesis:
\begin{align*}
    p(H_0 \mid y) &= p(\theta \in \Omega_0 \mid y) = \int_{\Omega_0} \pi(\theta \mid y) d\theta \\
    p(H_1 \mid y) &= p(\theta \in \Omega_1 \mid y) = \int_{\Omega_1} \pi(\theta \mid y) d\theta \\
    \span \text{Since } \Omega_0 \cap \Omega_1 = \emptyset, \text{ we have:} \\
    p(H_0 \mid y) &+ p(H_1 \mid y) = 1
\end{align*}

We choose the hypothesis with the \emph{highest} posterior probability.

\begin{example}{Hypothesis testing, 3 hypothesis}{}
    The time needed for a specific radioactive particle to disintegrate follows an exponential
    model. Physicists agree to use a Gamma$(\alpha=10,\,\beta=10)$ as a prior distribution.

    The Bayesian model is:
    \begin{align*}
        \tilde y \mid \lambda &\sim \exp(\lambda) \\
        \lambda &\sim \text{Gamma}(\alpha=10,\,\beta=10)
    \end{align*}
    We want to choose among the following hypotheses:
    \begin{alignat*}{1}
        H_0&:\; \lambda \in [0,0.5] \\
        H_1&:\; \lambda \in [0.5,1.5] \\
        H_2&:\; \lambda \in [1.5,\infty)
    \end{alignat*}

    The observed data is: $0.9,\,1.1,\,1$ which corresponds to a posterior distribution:
    \begin{equation*}
        \lambda \mid y \sim \text{Gamma}(\alpha=13,\,\beta=13)
    \end{equation*}

    We calculate the posterior distribution for each hypothesis:

    \begin{multicols}{2}
        \begin{align*}
            p(H_0 \mid y) &= \int_0^{0.5} \pi(\lambda \mid y) d\lambda \approxeq 0.016 \\
            p(H_1 \mid y) &= \int_{0.5}^{1.5} \pi(\lambda \mid y) d\lambda \approxeq 0.935 \\
            p(H_2 \mid y) &= \int_{1.5}^\infty \pi(\lambda \mid y) d\lambda \approxeq 0.049
        \end{align*}
        \vfill\null
        \columnbreak
        \null
        \begin{tikzpicture}[
            aoc/.style={draw=none,fill,opacity=0.3},
            ]
			\begin{axis}[
					samples at={0,0.01,...,1},
					width=0.50\textwidth,
					height=0.33\textwidth,
					ytick=\empty,
					xmin=0,
					xmax=2.5,
					xlabel=$\lambda$,
					legend pos=north east,
				]

				\addplot [aoc,fill=blue,forget plot] gnuplot [raw gnuplot] { \GnuplotDefs plot [x=0:0.5] gammaPdf(x,13,13); } \closedcycle;
				\addplot [aoc,fill=green,forget plot] gnuplot [raw gnuplot] { \GnuplotDefs plot [x=0.5:1.5] gammaPdf(x,13,13); } \closedcycle;
				\addplot [aoc,fill=red,forget plot] gnuplot [raw gnuplot] { \GnuplotDefs plot [x=1.5:2.5] gammaPdf(x,13,13); } \closedcycle;
				\addplot [mark=none,black,line width=1.25pt] gnuplot [raw gnuplot] {
				\GnuplotDefs
				plot [x=0:2.5] gammaPdf(x,13,13);
				}; \addlegendentry{$\pi(\lambda \mid y)$}
            \addlegendimage{area legend,aoc,fill=blue}; \addlegendentry{$H_0$}
            \addlegendimage{area legend,aoc,fill=green}; \addlegendentry{$H_1$}
            \addlegendimage{area legend,aoc,fill=red}; \addlegendentry{$H_2$}
			\end{axis}
             % \draw[decorate,decoration={brace,raise=1.9em}] (R) -- (L) node[midway,below=2.2em] {$CI_p$};
		\end{tikzpicture}
\end{multicols}
We choose the hypothesis with the highest posterior probability: $H_1$.
\end{example}

%! TEX root = **/000-main.tex

\section{Bayesian Computation}

In Bayesian statistics, we must always calculate the posterior
distribution $\pi(\theta \mid y)$:

\begin{equation*}
	\pi(\theta \mid y) = \frac{L_y(\theta)p(\theta)}{\int L_y(\theta)p(\theta)\,d\theta}
\end{equation*}

The integral in the denominator is usually intractable analytically. For example, if
$\theta$ is a vector $\left(\theta_1,\,\theta_2,\,\dots,\,\theta_p \right)$,
the integral is a multidimensional integral:

\begin{equation*}
	\int L_y(\theta)p(\theta)\,d\theta = \int_{\Omega_p} \int_{\Omega_{p-1}} \cdots \int_{\Omega_1} L_y(\theta)p(\theta)\,d\theta_1\,d\theta_2\,\dots\,d\theta_p
\end{equation*}

Therefore, we must resort to numerical methods that simulate values from the posterior
distribution.
We call these methods \iemph{Bayesian computation}.

\subsection{Markov Chain Monte Carlo}

The MCMC algorithms allows us to simulate values from the posterior distribution given
a Bayesian model and Data.

The most common MCMC algorithms are:
\begin{itemize}
	\item Gibbs Sampling
	\item Metropolis Hasting
	\item Hamiltonian Monte Carlo
\end{itemize}

Given an initial value $\theta^{(0)}$, the MCMC algorithms generate a sequence of successive values
$\theta^{(1)},\,\theta^{(2)},\,\dots,\,\theta^{(n)}$. After a burn-in period, the values
$\theta^{(n)}$ are assumed to be independent and identically distributed (i.i.d.) from the
posterior distribution $\pi(\theta \mid y)$ (the chain has converged).

We can use the values after the burn-in period as simulations from the posterior distribution.
Therefore, we need to make sure that the chain has converged.

\subsubsection{Checking the convergence}

The convergence check can be done graphically by starting two (or more) chains
from different initial values and plotting the values of the chain. If the
chains have converged, the values of the chains should be similar.

Additionally, we must check that there is no autocorrelation in the chain.

If the chains have not converged, we should increase the number of iterations.

\begin{figure}[H]
	\begin{tikzpicture}
		\begin{axis}[
				samples at={0,10,...,3000},
				width=0.8\textwidth,
				height=0.3\textwidth,
				ytick=\empty,
				xmin=0,
				xmax=3000,
				xlabel={Iteration},
				legend pos=north east,
                xtick={0,500,1000,1500,2000,2500,3000},
			]
            \addlegendimage{dashed,blue,line width=2pt};
            \addlegendentry{$B$};
            \addplot[mark=none,color=red,samples at={0,10,...,1000}]
                {rand*10 - x*0.02+50};
            \addplot[mark=none,color=red,samples at={1000,1010,...,3000}]
                {30 + rand*10};
            \addplot[mark=none,color=green,samples at={0,10,...,1000}]
                {rand*10 + x*0.03};
            \addplot[mark=none,color=green,samples at={1000,1010,...,3000}]
                {30 + rand*10};
            \draw[dashed,blue,line width=2pt] (1000,-100) -- (1000,100);
		\end{axis}
	\end{tikzpicture}
    \caption{Example of two chains that converge at iteration $B = 1000$}
\end{figure}

The point $B$ is the iteration where the chains converge. We should set the burn-in parameter
to a number greater than $B$.

\subsubsection{JAGS}

JAGS is a software package for Bayesian analysis that uses MCMC algorithms.
It is written in C++ and has an R interface.
It allows us to specify a Bayesian model using a simple language (BUGS) and
it automatically generates the code for the MCMC algorithms.

Additionally, we can set the starting points of the chains, the number of chains,
the number of iterations, the burn-in period and the thinning factor.
The thinning factor allows us to reduce the autocorrelation in the chain
by only keeping every $n$ values.

\begin{recap}{}{}
	\begin{itemize}
		\item MCMC algorithms allow us to simulate values from the posterior distribution
		      starting from the Bayesian model and the data.
		\item We \emph{must} check the convergence of the chains before making any inference.
	\end{itemize}
\end{recap}




%! TEX root = **/000-main.tex

\section{Hierarchical Models}

Hierarchical models treat the parameters of the \emph{prior} distribution
as random variables, adding a new layer of uncertainty to the model:

\begin{tikzpicture}
    \node[align=center] (non-hier) {
            Bayesian Model (non-hierarchical) \\[0.7em]
            $\begin{aligned}
                \tilde y &\sim p(\tilde y \mid \theta) \\
                \theta &\sim \pi(\theta)
            \end{aligned}$
        };
    \node[align=center, right=of non-hier] (hier) {
            Hierarchical Bayesian Model \\[0.7em]
            $\begin{aligned}
                \tilde y &\sim p(\tilde y \mid \theta) \\
                \theta &\sim \pi(\theta \mid \gamma ) \\
                \gamma &\sim \Psi(\gamma)
            \end{aligned}$
        };
\end{tikzpicture}

We call $\gamma$ the \iemph{hyperparameter} of the model and $\Psi(\gamma)$
the \iemph{hyperprior} distribution. We can have multiple layers of
hyperparameters and hyperpriors.

We can make inference and predictions at any level of the model.

\paragraph{Observation} we can convert a hierarchical model into a non-hierarchical
model by integrating out the hyperparameters. But then we lost the option
of making inference and predictions at the second level.

Further reading: Chapters 5 and 15 of \cite{gelman_bayesian_2013}

%! TEX root = **/000-main.tex

\section{Model Checking}

A model will be valid if it is able to generate data like the one we have observed.

The Bayesian model is a ``data simulator'' as it uses the prior distribution to generate data.

\paragraph{Steps}
\begin{enumerate}
    \item Chose a statistic(s)
    \item Compute its reference distribution under the model
    \item Compare the statistic with the reference distribution
\end{enumerate}

\subsection{Choosing a statistic}

Can be numeric $T(y=(y_1,\, y_2,\, \dots,\, y_n))$ or graphical and should summarize the data and focus
on the relevant aspects to our objective.

\subsection{Computing the reference distribution of the statistic under the model}

Using the posterior distribution, we will simulate values of the statistic to approximate
the reference distribution:

\begin{align*}
    \span \text{We will simulate replicas of the observed data set:} \\
    y^{rep} = (y_1^{rep},\, y_2^{rep},\, \dots,\, y_n^{rep}) \quad \text{for} \quad rep = 1,\, 2,\, \dots,\, M \\
    \span \text{For each replica we will calculate the statistic} \\
    T(y^{rep}) = T(y_1^{rep},\, y_2^{rep},\, \dots,\, y_n^{rep}) \\
    \span \text{These values will allow us to approximate the reference distribution} \\
    p(T(\tilde y) \mid y)
\end{align*}

If our objective is to make a prediction, a common way to validate the model is:
\begin{enumerate}
    \item Extract a subset of data
    \item Implement the model without the extracted data
    \item Compare the predictions the model makes with this extracted data.
        For example, whether the prediction intervals contain the observed data.
\end{enumerate}

\begin{definition}{Model Construction}{}
    The iterative process of proposing a model, validating it, identifying its limitations and
    perhaps proposing a new model is called \iemph{model construction}.
\end{definition}


\paragraph{Further reading:} Chapter 6 of \cite{gelman_bayesian_2013}


% Time series
\chapter{Time Series}
\section{Introduction}

\begin{definition}{Time series}{time-series}\index{time-series}
	Ordered sequence of observations of the same phenomenon. Typically
	measured at equally spaced successive instants of time.
	\begin{equation*}
		\{X_t\}_{t=1,\ldots,T} = \{X_1, X_2, \ldots, X_T\}
	\end{equation*}
\end{definition}

\subsection{Motivation}

Describing and forecasting time series is crucial in different areas of
knowledge; including finance, econometrics, signal processing and a long etc.

\subsection{Objectives}

\begin{itemize}
	\item \textbf{Description}: Describe temporal patterns in a time series: regular and/or
	      seasonal effects, cyclicity, trends, outliers, sudden changes, breaks, \dots
	\item \textbf{Estimation}: Estimate the values of the time series parameters.
	\item \textbf{Validation}: Validate the estimated parameters and decide if the estimated
	      parameters are significant or not.
	\item \textbf{Prediction/Forecasting}: Predict future values of the time series.
\end{itemize}

\begin{example}{AirPassengers}{AirPassengers}
	Monthly totals of international airline passengers in USA, 1949
	to 1960 (Box \& Jenkins, 1976).
	\begin{verbatim}
##      Jan Feb Mar Apr May Jun Jul Aug Sep Oct Nov Dec
## 1949 112 118 132 129 121 135 148 148 136 119 104 118
## 1950 115 126 141 135 125 149 170 170 158 133 114 140
## 1951 145 150 178 163 172 178 199 199 184 162 146 166
## 1952 171 180 193 181 183 218 230 242 209 191 172 194
## 1953 196 196 236 235 229 243 264 272 237 211 180 201
## 1954 204 188 235 227 234 264 302 293 259 229 203 229
## 1955 242 233 267 269 270 315 364 347 312 274 237 278
## 1956 284 277 317 313 318 374 413 405 355 306 271 306
## 1957 315 301 356 348 355 422 465 467 404 347 305 336
## 1958 340 318 362 348 363 435 491 505 404 359 310 337
## 1959 360 342 406 396 420 472 548 559 463 407 362 405
## 1960 417 391 419 461 472 535 622 606 508 461 390 432
\end{verbatim}
	\begin{nscenter}
		\begin{tikzpicture}
			\begin{axis}[
					width=0.9\textwidth,
					height=0.4\textwidth,
					date coordinates in=x,
					table/col sep=comma,
					xlabel=Time,
					ylabel=Passengers,
					xticklabel=\year,
					xtick distance=366*2,
					mark size=1pt,
					mark=o,
					xmajorgrids=true,
					minor x tick num=1,
					xminorgrids=true,
				]
				\addplot table[x=Month,y=Passengers] {data/AirPassengers.csv};
			\end{axis}
		\end{tikzpicture}
	\end{nscenter}
\end{example}

\subsection{Exploratory Data analysis}

Plot of the series and identification of the components:

\begin{definition}{Trend ($T_t$)}{trend}\index{trend}
	Long term tendency of the series.

	Moving average of order $s$:
	\begin{equation*}
		T_t = \frac{1}{s} \sum_{i=1}^s X_{t-s/2+i}
	\end{equation*}
\end{definition}

\begin{definition}{Seasonal ($S_t$)}{seasonal}\index{seasonal}
	Pattern repeated periodically with the same period.

	\index{detrended series}
	\paragraph{Seasonal index} Mean for each period of detrended series ($X_t - T_t$).
\end{definition}

\begin{definition}{Cycle ($C_t$)}{cycle}\index{cycle}
	Pattern repeated periodically with \emph{non-constant} period.

	\vspace{1em}
	\begin{marker}
		Not easy to model due to the changing period.
	\end{marker}
\end{definition}

\begin{definition}{Random ($w_t$)}{random}\index{random}
	Random noise.

	\paragraph{Remainder} ($X_t - T_t - S_t - C_t$)
\end{definition}

Our \emph{goal} is to find a mathematical model that reflects the behavior of the observed
data.

\subsection{Modeling}

\subsubsection{Additive model}
We add the different components:\index{additive model}
\begin{equation}
	X_t = T_t + S_t + C_t + w_t \tag{additive}
\end{equation}

\begin{figure}[H]
	\begin{tikzpicture}
		\begin{axis}[
				width=0.9\textwidth,
				height=0.5\textwidth,
				date coordinates in=x,
				xticklabel=\year,
				xtick distance=366*2,
				xlabel=Time,
				ylabel=Passengers,
				ymajorgrids=true,
				yminorgrids=true,
				xmajorgrids=true,
				minor x tick num=1,
				xminorgrids=true,
				legend pos=north west,
			]
			\directlua{dofile("lua/decomposition.lua")}

			\addlegendentry{$X$ (data)}
			\addlegendentry{$T$ (trend)}
			\addlegendentry{$S$ (seasonal)}
			\addlegendentry{$T+S$}
			\addlegendentry{$w = X-T-S$ (random)}
		\end{axis}
	\end{tikzpicture}
	\caption{Decomposition of the \texttt{AirPassengers} series from example~\ref{example:AirPassengers}}
\end{figure}

\subsubsection{Deterministic model}
The expected value of $X_t$ depends on a parametric function $F$ of $t$ and the random
component does not depend on the previous values.
\begin{equation}
	X_t = F(t) + Z_t \qquad Z_t \sim N(0,\, \sigma_z^2) \tag{deterministic}
\end{equation}

\subsubsection{Stochastic model}
The expected value of $X_t$ depends on the previous values $X_{t-1}, X_{t-2}, \dots$ and/or
the previous random components $Z_{t-1}, Z_{t-2}, \dots$ plus a random component independent
of the past $Z_t$.
\begin{equation}
	X_t = G(X_{t-1}, X_{t-2}, \dots, Z_{t-1}, Z_{t-2}, \dots) + Z_t \qquad Z_t \sim N(0,\, \sigma_z^2) \tag{stochastic}
\end{equation}

\subsection{Box-Jenkins methodology}

\begin{nscenter}
	\begin{tikzpicture}[
			box/.style={draw, rectangle, minimum width=5cm, minimum height=1.5cm, align=center},
			decision/.style={diamond, minimum width=2cm, minimum height=1cm, align=center},
			box_l/.style={align=left},
		]
		\node[box, rounded corners] (ti) {Tentative\\Identification};
		\node[box, below=of ti] (est) {Estimation};
		\node[box, below=of est] (val) {Diagnostic\\ Checking};
		\node[box, decision, below=of val] (dec) {Model\\ok?};
		\node[box, rounded corners, below=of dec] (fin) {Final Model};

		\node[box_l, right=of ti] {Time Series Plot \\Range-Mean Plot\\ACF and PACF};
		\node[box_l, right=of est] {Least Squares or\\Maximum Likelihood};
		\node[box_l, right=of val] {Residual Analysis\\and Forecasts};
		\node[box_l, right=of fin] {Forecasting\\Explanation};


		\draw[->] (ti) -- (est);
		\draw[->] (est) -- (val);
		\draw[->] (val) -- (dec);
		\draw[->] (dec) edge[edge label={Yes}] (fin);

        \coordinate (fb) at ($(val.west)+(-3em,0)$);
        \node (no) at ($(dec.west)+(-2em,0)$) {No};
        \draw[->] (dec.west) -- (no) -| (fb) |- (ti.west);
	\end{tikzpicture}
\end{nscenter}

\subsection{Distribution of a general stochastic process}

First and second moments for the multivariate distribution of $\{X_t\}_{t=1\dots T}$:
\begin{align*}
	\mathbb{E}[(X_1, \dots, X_T)] & = (\mu_1, \dots, \mu_T)                                              \\
	Var[(X_1, \dots, X_T)]        & = \begin{pmatrix}
		                                  \sigma_1^2   & \sigma_{1,2} & \sigma_{1,3} & \dots  & \sigma_{T,1} \\
		                                  \sigma_{2,1} & \sigma_2^2   & \sigma_{2,3} & \dots  & \sigma_{T,2} \\
		                                  \sigma_{3,1} & \sigma_{3,2} & \sigma_3^2   & \dots  & \sigma_{T,3} \\
		                                  \vdots       & \vdots       & \vdots       & \ddots & \vdots       \\
		                                  \sigma_{T,1} & \sigma_{T,2} & \sigma_{T,3} & \dots  & \sigma_T^2
	                                  \end{pmatrix}
\end{align*}

Parameters of the model:
\begin{itemize}
    \item $T$ values for the mean $\mathbb{E}(X_t) = \mu_t$
    \item $T$ values for the variances $Var(X_t) = \sigma_t^2$
    \item $T(T-1)$ values for the covariances $Cov(X_t, X_s) = \sigma_{t,s}$
\end{itemize}

\subsection{Stationary Series}

Strict Stationary Process or Series has the following properites:
The joint distribution of the whole series does not depend on the time
origin:
\begin{equation}
    F_{X_1, \dots, X_t}(x_1, \dots, x_t) = F_{X_{1+s}, \dots, X_{t+s}}(x_{1+s}, \dots x_{t+s}) \forall t,s
\end{equation}

\subsection{Weakly Stationary process}
The two first moments of the multivariate distribution of the whole series
does not depend on the time origin:
\begin{itemize}
    \item Constant mean $\mu$.
    \item Constant variance $\sigma^2$.
    \item Constant auto-covariance structure $\sigma_{t,s} = \sigma_{\lVert t - s \rVert}$.
    \item The latter refers to the covariance between $X_t$ and $X_{t-1}$ being equal to 
        the covariance between $X_{t-s}$ and $X_{t-s-1}$.
\end{itemize}

Weakly Stationary process + Gaussian multivariate distribution $\Longrightarrow$ Strict Stationary process.

\begin{question}{Is our data stationary? How can we detect?}{}
    \begin{itemize}
        \item Plot the data.
        \item Identify no stationary components (trends, seasonal patterns, cycles).
        \item Transform the series to remove those components.
        \item For the transformed (stationary) series, plot and analyze the sample autocorrelation.
    \end{itemize}
\end{question}

\begin{marker}
    It is very common that the variance of the series increases when the level of the series rises.
\end{marker}

\begin{figure}[H]
    \begin{tikzpicture}
        \begin{axis}[
                width=0.48\textwidth,
                height=0.4\textwidth,
                date coordinates in=x,
                table/col sep=comma,
                xlabel=Time,
                ylabel=Passengers,
                xticklabel=\year,
                xtick distance=366*4,
                mark size=1pt,
                mark=o,
                xmajorgrids=true,
                minor x tick num=1,
                xminorgrids=true,
                title={\bfseries Non-Constant Variance},
            ]
            \addplot table[x=Month,y=Passengers] {data/AirPassengers.csv};
        \end{axis}
    \end{tikzpicture}
    \begin{tikzpicture}
        \begin{axis}[
                width=0.48\textwidth,
                height=0.4\textwidth,
                table/col sep=comma,
                xlabel=Time,
                ylabel=$\text{CO}_2$ (ppm),
                mark size=1pt,
                mark=o,
                xmajorgrids=true,
                minor x tick num=1,
                xminorgrids=true,
                enlargelimits=false,
                x tick label style={/pgf/number format/.cd, set thousands separator={}},
                title={\bfseries Constant Variance},
            ]
            \addplot+[red] table[mark=none,x=decimal date,y=average] {./data/co2_mm_mlo.csv};
        \end{axis}
    \end{tikzpicture}
    \caption{Examples of constant and non-constant variance}
\end{figure}

\subsection{Tools to diagnose the non-constant variance}

\paragraph{Mean-Variance Plot} Calculate the mean and variance of consecutive
groups of 8-12 observations. Plot the variance against the mean of each group.

\begin{figure}[H]
	\begin{tikzpicture}
		\begin{axis}[
				width=0.9\textwidth,
				height=0.5\textwidth,
				xlabel=Mean,
				ylabel=Variance,
				% ymajorgrids=true,
				% yminorgrids=true,
				% xmajorgrids=true,
			]
			\directlua{dofile("lua/variance-mean.lua")}
		\end{axis}
	\end{tikzpicture}
	\caption{Mean-Variance plot of the \texttt{AirPassengers} series}
\end{figure}

\paragraph{Boxplot for periods} Represent the boxplot for each group of 8-12
observations. The height of the box (IQR) is a robust estimate of variability.

\begin{figure}[H]
	\begin{tikzpicture}
		\begin{axis}[
				width=0.9\textwidth,
				height=0.5\textwidth,
				xlabel=Time,
				ylabel=Passengers,
				% ymajorgrids=true,
				% yminorgrids=true,
				% xmajorgrids=true,
                boxplot/draw direction=y,
                xtick={2,4,6,8,10,12},
                xticklabels={1950,1952,1954,1956,1958,1960},
                % boxplot/median/.style={line width=2pt},
			]
			\directlua{dofile("lua/boxplot.lua")}
		\end{axis}
	\end{tikzpicture}
	\caption{Boxplot for periods of the \texttt{AirPassengers} series}
\end{figure}

\begin{itemize}
    \item If the variance is similar for all the groups $\Longrightarrow$ No scale transformation.
    \item If the variance is higher for higher values of the mean $\Longrightarrow$ Change the scale.
\end{itemize}

\subparagraph{Box-Cox transformation}
\begin{equation}
    Y_t = \begin{cases}
        \frac{X_t^\lambda - 1}{\lambda} & \text{if } \lambda \neq 0 \\
        \ln(X_t) & \text{if } \lambda = 0
    \end{cases} \qquad \lambda \in [-1,\,2]
\end{equation}
\begin{marker}
    Usually, the logarithmic transformation is used since it easier to interpret.
\end{marker}


\section{ARMA Models}
% 2022-11-15

\begin{definition}{Stationary process (weakly stationary)}{}
    
    Gaussian process $\{X_t\},\,t=1,/ldots,n$
    \begin{align*}
        X &= (X_t) \sim \text{Multivariate Gaussian/Normal distribution} \\
        E(X) &= \mu = \left( \mu_1, \ldots, \mu_n \right)' \\
    \end{align*}

    The Covariance matrix $(n \times n)$ is given by:
    \begin{equation*}
        V(X) = \Gamma = \left\{ \gamma(t_i,t_j) \mid t,j = 1,\ldots,n\right\}
    \end{equation*}

    And the multivariate Normal density function can be written as:
    \begin{equation*}
        f(x) = (2\pi)^{-n/2} |\Gamma|^{-1/2} \text{exp}\left\{ -\frac{1}{2}(x-\mu)' \Gamma^{-1} (x-\mu) \right\}
    \end{equation*}
    where $|\Gamma|$ is the determinant of $\Gamma$.
\end{definition}

\begin{definition}{General Stochastic model for a time series}{}
    \begin{equation*}
        X_t = G\left(
            X_{t-1},\,X_{t-2},\,\ldots,\,Z_{t-1},\,Z_{t-2},\,\ldots
        \right) + Z_t, \quad Z_t \sim \mathcal{WN}(\sigma^2)
    \end{equation*}
    where $G$ is a deterministic function and $Z_t$ is a white noise process.
\end{definition}

\begin{definition}{Linear Stochastic model for a time series}{}
    \begin{align*}
        X_t = \sum_{i=1}^p \phi_i X_{t-i} + \sum_{j=1}^q \theta_j Z_{t-j} + Z_t, &&Z_t \sim \mathcal{WN}(\sigma^2) \\
        && \phi_1,\,\ldots,\,\phi_p,\,\theta_1,\,\ldots,\,\theta_q \in \mathbb{R}
    \end{align*}
    where $Z_t$ is a white noise process.

    The variable at time $t$ is a \iemph{linear combination} of past observations
    and disturbances plus a new disturbance independent from the past.
\end{definition}

\begin{definition}{$p$-th order Auto-Regressive Model (AR($p$))}{}
    \begin{align*}
        X_t &= \sum_{i=1}^p \phi_i X_{t-i} + Z_t \\
        Z_t &= \left(
            1 - \sum_{i=1}^p \phi_i B^i
            \right) X_t
    \end{align*}
    where $B$ is the backward shift operator.

    \tcblower

    An AR($p$) model is a regression model with lagged values of the dependent
    variable in the independent variable positions, hence the name Auto-Regressive.
\end{definition}






\begin{definition}{Stochastic processes (ARMA)}{}
	\begin{equation*}
		\text{ARMA}(p,q) \equiv \text{AR}(p) + \text{MA}(q)
	\end{equation*}
	\begin{equation*}
		X_t = \underbrace{
			\sum_{i=1}^p \phi_i X_{t-i}
		}_{\text{AR}(p)}
		+ \underbrace{
			\sum_{i=1}^q \theta_i Z_{t-i}
		}_{\text{MA}(q)}
		+ \underbrace{
			Z_t
			\vphantom{\sum_{i=1}^q}
		}_{ \substack{\text{Random} \\ \text{component}} }
		\quad \text{where} \; Z_t \sim \mathcal{N}(0,\,\sigma^2)
	\end{equation*}
	\begin{equation*}
		\text{ARMA}(p,q) \rightarrow \Phi_p(B) X_t = \Theta_q(B) Z_t
	\end{equation*}
\end{definition}


% TODO: idk
\begin{alignat*}{3}
	AR & (1) & \quad \rho(h) & = \phi^h                                       & \quad X_t   & = \phi X_{t-1} + Z_t \\
	AR & (p) & \quad \rho(h) & = \phi_1 \rho(h-1) + \cdots + \phi_p \rho(h-p) & \quad h > p
\end{alignat*}

\begin{alignat*}{3}
	MA & (1) & \quad \rho(h) & = \begin{cases}
		                             0                       & h > 1            \\
		                             \frac{\phi}{1-\theta^2} & \text{otherwise}
	                             \end{cases}                 \\
	MA & (q) & \quad \rho(h) & = \phi_1 \rho(h-1) + \cdots + \phi_p \rho(h-p) & \quad h > p
\end{alignat*}

We have a decreasing pattern (damped oscillations) for $h > p$ with
an infinite number of oscillations.

The last lag different from zero on the ACF of MA gives us the $q$.

PACF and ACF are complementary on AR and MA processes.

\begin{equation*}
	V(\hat \rho (h))  \approxeq \frac{1}{T}
\end{equation*}
\begin{equation*}
	s(\hat \rho (h))  = \sqrt{V(\hat \rho (h))} \approxeq \frac{1}{\sqrt{T}}
\end{equation*}

\begin{align*}
	H_0 & : \hat \rho (h) = 0    \\
	H_1 & : \hat \rho (h) \neq 0
\end{align*}

If the estimation lies within the confidence interval
$\hat \rho (h) \in \left[-\frac{1.96}{\sqrt{T}}, \frac{1.96}{\sqrt{T}}\right]$,
we cannot reject the null hypothesis $H_0$.

\begin{note}
	We cannot always rely on this confidence interval.

	Remember that what we care about is which is the last one which is not
	equal to zero.

	We should only apply it for the initial lags, not for lags greater
	than 0 that are well in the past.
\end{note}

\begin{marker}
	Common mistake: if we identify in a plot a AR or MA, we don't have an
	ARMA (AR(p) implies MA(infinity) and vice versa).

	We cannot find the p, and q for ARMA(p, q) from the ACF and PACF.
\end{marker}

\begin{figure}[H]
	\begin{tikzpicture}
		\begin{axis}[
				ymin=-1, ymax=1,
				enlargelimits=false,
				width=0.45\textwidth,
				title=ACF,
				xlabel=$h$,
				ylabel=ACF,
			]
			% set pgf math constant
			\pgfmathsetmacro{\T}{200}
			\pgfmathsetmacro{\mx}{60}
			\pgfmathsetmacro{\ci}{1.96/sqrt(\T)}
			\addplot[blue, dashed, mark=none, samples=2, domain=0:\mx] {\ci};
			\addlegendentry{95\% CI};
			\addplot[blue, dashed, mark=none, samples=2, domain=0:\mx] {-\ci};
			\addplot[black, mark=none, samples=2, domain=0:\mx] {0};
			\addplot[ycomb,samples=\mx, domain=0:\mx, mark=none] {2.5*cos(\x*15)/(\x+2.5) + rand*\ci/2};
			% \addplot[ycomb, domain=0:\mx, mark=none] {cos(\x*15)};
		\end{axis}
	\end{tikzpicture}
	\hfill
	\begin{tikzpicture}
		\begin{axis}[
				ymin=-1, ymax=1,
				enlargelimits=false,
				width=0.45\textwidth,
				title=PACF,
				xlabel=$h$,
				ylabel=PACF,
			]
			% set pgf math constant
			\pgfmathsetmacro{\T}{200}
			\pgfmathsetmacro{\mx}{60}
			\pgfmathsetmacro{\ci}{1.96/sqrt(\T)}
			\addplot[blue, dashed, mark=none, samples=2, domain=0:\mx] {\ci};
			\addlegendentry{95\% CI};
			\addplot[blue, dashed, mark=none, samples=2, domain=0:\mx] {-\ci};
			\addplot[black, mark=none, samples=2, domain=0:\mx] {0};
			\addplot[ycomb,samples=\mx-3, domain=3:\mx, mark=none] {rand*\ci};
			\addplot[ycomb] coordinates {
					(0, 0)
					(1, 0.8)
					(2, 0.3)
				};
		\end{axis}
	\end{tikzpicture}
	\caption{ACF and PACF}
\end{figure}

\subsection{ARIMA}

ARIMA(p,d,q)

AR = auto correlation
I =
MA = Moving average

significant test

% t  = phi/s

% H_0 = phi_i = 0

% if t is > 2 then H_0 rejected

which means that phi must be kept in the value.

(is the coeficient is more than two times the absolute value)


% \nocite{*}

\printindex
\vfill
\printbibliography

\end{document}
