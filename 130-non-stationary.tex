%! TEX root = **/000-main.tex
\section{Non-stationary processes}
% 2022-11-22

\begin{align*}
	MA(1) & = \theta_1 Z_{t-1} + Z_t
\end{align*}

\begin{align*}
	MA(1)              & \,\text{invertible}                     \\
	                   & \Updownarrow                            \\
	1 + \theta_1 B = 0 & \implies |B| = |\frac{1}{\theta_1}| > 1 \\
	\span\boxed{|\theta_1| < 1}
\end{align*}

\begin{align*}
	AR(1) & : (1 -\phi_1 B) X_t = Z_t
\end{align*}

\begin{align*}
	AR(1) & \,\text{causal}                              \\
	      & \Updownarrow                                 \\
	1 - \phi B = 0 \implies |B| = |\frac{1}{\phi_1}| < 1 \\
	\boxed{|\phi_1| < 1}
\end{align*}

ARMA(2,2):

\begin{align*}
	\underbrace{(1 + \phi_1B + \phi_2B^2)}_{AR(2)}
	X_t
	=
	\underbrace{(1 + \theta_1B + \theta_2B^2)}_{MA(2)}
	Z_t \\
	B = \frac{
		-\theta_1 \pm \sqrt{\theta_1^2 - 4\theta_2}
	}{
		2\theta_2
	}
\end{align*}

\begin{note}
	In general if the degree is greater than 2 we
	can only solve it with computer.
\end{note}

\begin{note}
	$AR(p)$ is always invertible.
\end{note}

\begin{note}
	\begin{align*}
		MA(q) \implies X_t = \Theta_q(B) Z_t
	\end{align*}
	All $MA(q)$ models are causal.
\end{note}

\begin{example}{}{}
	\begin{align*}
		X_t                   & = 1.5 + 0.36X_{t-2} + Z_t \tag{AR(2)} \\
		(1 - 0.36B^2) X_t = 1.5 + Z_t                                 \\
		A(\infty) \begin{cases}
			          \pi_1 & = 0                    \\
			          \pi_2 & = -0.36                \\
			          \pi_k & = 0 \text{ for } k > 2
		          \end{cases}                      \\[2em]
		1 - 0.36B^2 = 0       & \rightarrow{} B = \pm \sqrt{1}{0.36}
		= \pm \frac{5}{3}                                             \\
		|B| = \frac{5}{3} > 1 & \implies{} \text{CAUSAL}
	\end{align*}

	\tcbline

	\begin{align*}
		X_t - X_{t-1} + X_{t-2} & = Z_t - Z_{t-2} \tag{ARMA(2,1)}               \\
		(1 - B + B^2) X_t       & = (1 - B^2) Z_t                               \\
		1 - B + B^2 = 0         & \rightarrow{} B = \frac{-1\pm\sqrt{1 - 4}}{2}
		= 1/2 \pm \frac{\sqrt{3}}{2}i                                           \\
		|B| = \sqrt{\left(\frac{1}{2}\right)^2 + \left(\frac{\sqrt{3}}{2}\right)^2}
		= 1                                                                     \\
	\end{align*}

	Since they both have $|B| = 1$ non of them are
	invertible (they must be strictly greater than 1).

	\begin{figure}[H]
		\begin{tikzpicture}
			\begin{axis}[
					axis lines=middle,
					xmin=-1.5, xmax=1.5,
					ymin=-1.5, ymax=1.5,
					% fixed aspect ratio
					axis equal image,
				]
				% unit circle
				\addplot[domain=0:360, samples=60, smooth, color=black, dashed]
				({cos(x)}, {sin(x)});

				\draw[blue] (0, 0) -- (0.5, 0.866025);
				\draw[blue] (0, 0) -- (0.5, -0.866025);
				\addplot[blue,mark=*,only marks, point meta=explicit symbolic,
					nodes near coords,
				] coordinates {
						(0.5, 0.866025)[$\left(\frac{1}{2},\, \frac{\sqrt{3}}{2} \right)$]
						(0.5, -0.866025)[$\left(\frac{1}{2},\, -\frac{\sqrt{3}}{2} \right)$]
					};
			\end{axis}
		\end{tikzpicture}
	\end{figure}

\end{example}

\subsection{Causality}

A time series $X_t$ that has a unit root $(1 - B)^d$
\begin{align*}
	X_t \sim \underbrace{I(d)}_{Integrated} \iff (1 - B)^d X_t
	\tag{Stationary (Casual)}
\end{align*}

From this we devise the AR\textbf{I}MA(p,d,q) model:

\begin{definition}{AutoRegressive Integrated Moving Average (ARIMA)}{}
	\begin{equation*}
		\text{ARIMA}(p,\,d,\,q) \longrightarrow{} \underbrace{\Phi_p(B)}_{AR(p)}
		\underbrace{(1-B)^d}_{I(d)}
		X_t =
		\underbrace{\Theta_q(B)}_{MA(q)}
		Z_t
		% \tag{ARIMA}
	\end{equation*}
\end{definition}

\subsection{Identification}

We transform $X_t \longrightarrow{} W_t$ stationary.
From this we get the value of $d$. Which is the
number of regular differences we have applied to obtain
stationarity.

Sample ACF/PACF from $W_t \implies \text{ARMA}(p,\,q)$.

\begin{definition}{Random Walk}{}

	$\text{ARIMA}(0,\,1,\,0)$, it is just the previous
	value plus a random error.

	\begin{figure}[H]
		\begin{tikzpicture}
			\begin{axis}
				\addplot[mark=none] coordinates{
						\directlua{
							v = 0
							for i = 1, 300 do
							v = v + math.random()*10 - 5
							tex.print("(" .. i .. ", " .. v .. ")")
							end
						}
					};
			\end{axis}
		\end{tikzpicture}
	\end{figure}

\end{definition}

\section{SARIMA}

We combine the regular and seasonal dependence into the model,
from that we obtain the SARIMA model.

\begin{definition}{Regular dependence}{}
	How $X_t$ depends on $X_{t-1}$, $X_{t-2}$, etc.
\end{definition}

\begin{definition}{Seasonal dependence}{}
	How $X_t$ depends on $X_{t-s}$, $X_{t-2s}$, etc.

	Present when we have a seasonal pattern.
\end{definition}

\begin{definition}{Seasonal ARIMA (SARIMA)}{}
	\begin{align*}
		\text{SARIMA}
		\underbrace{(p,\,d,\,q)}_{\text{Regular}}
		\underbrace{\left(P,D,Q\right)_s}_{\text{Seasonal}} \\
		%
		\downarrow                                          \\
		%
		\underbrace{\phi_p(B)}_{AR(p)}
		\overbrace{\Phi_P(B^s)}^{AR(P)_s}
		\underbrace{(1-B)^d}_{I(d)}
		\overbrace{(1-B^s)^D}^{I(D)_s}
		X_t =
		\underbrace{\theta_q(B)}_{MA(q)}
		\overbrace{\Theta_Q(B^s)}^{MA(Q)_s}
		Z_t
	\end{align*}
\end{definition}


\subsection{Identification}

\begin{figure}[H]
	\begin{tikzpicture}[
			every node/.style={rectangle},
			node distance=2cm,
		]
		\node (start) {$X_t$};
		\node[below right of=start] (log) {$\log X_t$};
		\node[below of=log] (sdiff) {$(1-B^s)X_t$};
		\node[below of=sdiff] (diff) {$(1-B)X_t\;$};
        \node[right=of diff,align=left] (leg_diff) {$d$ = number of\\ regular differences};
		\coordinate[below left of=diff] (end_prev);
        \node[below of=end_prev,node distance=0.5cm] (end) {$W_t$};

		\draw[->] ($(start |- log.north) + (0, 5mm)$) -| (log.north);
		\draw[->] (log.south) |- ($(start |- log.south) + (0, -3mm)$);

		\draw[->] ($(start |- sdiff.north) + (0, 5mm)$) -| (sdiff.north);
		\draw[->] (sdiff.south) |- ($(start |- sdiff.south) + (0, -3mm)$);

		\draw[->] ($(start |- diff.north) + (0, 5mm)$) -| (diff.north);
		\draw[->] (diff.south) |- ($(start |- diff.south) + (0, -3mm)$);

		\draw[->] (start) -- (end);
	\end{tikzpicture}
\end{figure}

\begin{example}{SARIMA}{}

\end{example}
