%! TEX root = **/000-main.tex

\section{Model Checking}

A model will be valid if it is able to generate data like the one we have observed.

The Bayesian model is a ``data simulator'' as it uses the prior distribution to generate data.

\paragraph{Steps}
\begin{enumerate}
    \item Chose a statistic(s)
    \item Compute its reference distribution under the model
    \item Compare the statistic with the reference distribution
\end{enumerate}

\subsection{Choosing a statistic}

Can be numeric $T(y=(y_1,\, y_2,\, \dots,\, y_n))$ or graphical and should summarize the data and focus
on the relevant aspects to our objective.

\subsection{Computing the reference distribution of the statistic under the model}

Using the posterior distribution, we will simulate values of the statistic to approximate
the reference distribution:

\begin{align*}
    \span \text{We will simulate replicas of the observed data set:} \\
    y^{rep} = (y_1^{rep},\, y_2^{rep},\, \dots,\, y_n^{rep}) \quad \text{for} \quad rep = 1,\, 2,\, \dots,\, M \\
    \span \text{For each replica we will calculate the statistic} \\
    T(y^{rep}) = T(y_1^{rep},\, y_2^{rep},\, \dots,\, y_n^{rep}) \\
    \span \text{These values will allow us to approximate the reference distribution} \\
    p(T(\tilde y) \mid y)
\end{align*}

If our objective is to make a prediction, a common way to validate the model is:
\begin{enumerate}
    \item Extract a subset of data
    \item Implement the model without the extracted data
    \item Compare the predictions the model makes with this extracted data.
        For example, whether the prediction intervals contain the observed data.
\end{enumerate}

\begin{definition}{Model Construction}{}
    The iterative process of proposing a model, validating it, identifying its limitations and
    perhaps proposing a new model is called \iemph{model construction}.
\end{definition}


\paragraph{Further reading:} Chapter 6 of \cite{gelman_bayesian_2013}
