%! TEX root = **/000-main.tex

\section{Hierarchical Models}

Hierarchical models treat the parameters of the \emph{prior} distribution
as random variables, adding a new layer of uncertainty to the model:

\begin{tikzpicture}
    \node[align=center] (non-hier) {
            Bayesian Model (non-hierarchical) \\[0.7em]
            $\begin{aligned}
                \tilde y &\sim p(\tilde y \mid \theta) \\
                \theta &\sim \pi(\theta)
            \end{aligned}$
        };
    \node[align=center, right=of non-hier] (hier) {
            Hierarchical Bayesian Model \\[0.7em]
            $\begin{aligned}
                \tilde y &\sim p(\tilde y \mid \theta) \\
                \theta &\sim \pi(\theta \mid \gamma ) \\
                \gamma &\sim \Psi(\gamma)
            \end{aligned}$
        };
\end{tikzpicture}

We call $\gamma$ the \iemph{hyperparameter} of the model and $\Psi(\gamma)$
the \iemph{hyperprior} distribution. We can have multiple layers of
hyperparameters and hyperpriors.

We can make inference and predictions at any level of the model.

\begin{note}
    We can convert a hierarchical model into a non-hierarchical
model by integrating out the hyperparameters. But then we lost the option
of making inference and predictions at the second level.
\end{note}

Further reading: Chapters 5 and 15 of \cite{gelman_bayesian_2013}
