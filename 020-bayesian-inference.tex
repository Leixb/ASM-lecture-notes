%! TEX root = **/000-main.tex

\section{Bayesian Inference}

\subsection{Posterior distribution as an estimator}

The posterior distribution is a compromise between the prior
distribution (the information before observing the data) and the
likelihood function (the data information).

The posterior distribution $\pi(\theta \mid y)$ is the natural
Bayesian estimator for $\theta$ given the data $y$.

\subsection{Point Estimation}
Any measure of the location of $\pi(\theta \mid y)$ will serve as a point
estimate:
\begin{align*}
    \hat{\theta}_{pe} &= E(\theta \mid y) = \int \theta \pi(\theta \mid y) \;\partial\theta \\
    \hat{\theta}_{pme} &\text{ is such that } \int_{-\infty}^{\hat{\theta}_{pme}} \pi(\theta \mid y) \;\partial\theta = \frac{1}{2} \\
    \hat{\theta}_{pmo} &= \argmax_{\theta} \pi(\theta \mid y) \\
\end{align*}

This can also be computed numerically by simulation. Let $\theta^{(1)}, \dots , \theta^{(M)}$ be
simulations of $\theta$ from $\pi(\theta \mid y)$, then:
\begin{equation}
    \hat{\theta}_{pe} = \frac{1}{M} \sum_{i=1}^M \theta^{(i)}
\end{equation}

\paragraph{Observation} We can also define a prior point estimator using the prior distribution
instead of the posterior distribution.

\subsection{Interval Estimation}
A posterior credibility (or probability) interval of $p$ for $\theta$, $IC_p$ is
any region of $\Omega$ such that:
\begin{equation}
    p(\theta \in IC_p \mid y) = \int_{IC_p} \pi(\theta \mid y) \;\partial\theta = p
\end{equation}
Usually we use intervals based on percentiles.

\begin{example}{Confidence interval}{}
	\begin{center}
		\begin{tikzpicture}
			\begin{axis}[
					samples at={0,0.01,...,1},
					width=0.8\textwidth,
					height=0.33\textwidth,
					ytick=\empty,
					xmin=0,
					xmax=5,
					xlabel=$\theta$,
					legend pos=north west,
				]
				\addplot [mark=none,black] gnuplot [raw gnuplot] {
				\GnuplotDefs
				plot [x=0:5] beta(x/5,9,5);
				}; \addlegendentry{$\pi(\theta \mid y)$}
				\addplot [mark=none,fill=blue,opacity=0.3,forget plot]gnuplot [raw gnuplot] {
				\GnuplotDefs
				plot [x=2:4.3] beta(x/5,9,5);
				} \closedcycle;
            \addlegendimage{area legend,fill=blue,opacity=0.3}; \addlegendentry{$p$}
            \path (2,0) coordinate (L) (4.3,0) coordinate (R);
			\end{axis}
            \draw[decorate,decoration={brace,raise=1.9em}] (R) -- (L) node[midway,below=2.2em] {$CI_p$};
		\end{tikzpicture}
	\end{center}
    \begin{note}
        For the CI to be valid, the two tails must have equal area $(1-p)/2$.
    \end{note}
\end{example}

\subsection{Prediction}

We use the posterior predictive distribution to make predictions about future observations.
See definition~\ref{def:post-pred}.

\subsection{Hypothesis Testing}

Given a Bayesian model: $M = \{p(\tilde \theta \mid \theta),\,\theta \in \Omega\},\; \pi(\theta)$,
we split the parameter space into two disjoint regions: $\Omega \equiv \Omega_0 \cup \Omega_1$
and we want to decide to which subspace $\Omega_0$ or $\Omega_1$ the parameter $\theta$ belongs.
We formulate the hypothesis as:
\begin{align*}
    H_0:\; \theta \in \Omega_0 \\
    H_1:\; \theta \in \Omega_1
\end{align*}

After observing the data $y$, we can compute the posterior distribution for each hypothesis:
\begin{align*}
    p(H_0 \mid y) &= p(\theta \in \Omega_0 \mid y) = \int_{\Omega_0} \pi(\theta \mid y) \;d\theta \\
    p(H_1 \mid y) &= p(\theta \in \Omega_1 \mid y) = \int_{\Omega_1} \pi(\theta \mid y) \;d\theta \\
    \span \text{Since } \Omega_0 \cap \Omega_1 = \emptyset, \text{ we have:} \\
    p(H_0 \mid y) &+ p(H_1 \mid y) = 1
\end{align*}

We choose the hypothesis with the \emph{highest} posterior probability.

\begin{example}{Hypothesis testing, 3 hypothesis}{}
    The time needed for a specific radioactive particle to disintegrate follows an exponential
    model. Physicists agree to use a Gamma$(\alpha=10,\,\beta=10)$ as a prior distribution.

    The Bayesian model is:
    \begin{align*}
        \tilde y \mid \lambda &\sim \exp(\lambda) \\
        \lambda &\sim \text{Gamma}(\alpha=10,\,\beta=10)
    \end{align*}
    We want to choose among the following hypotheses:
    \begin{alignat*}{1}
        H_0&:\; \lambda \in [0,0.5] \\
        H_1&:\; \lambda \in [0.5,1.5] \\
        H_2&:\; \lambda \in [1.5,\infty)
    \end{alignat*}

    The observed data is: $0.9,\,1.1,\,1$ which corresponds to a posterior distribution:
    \begin{equation*}
        \lambda \mid y \sim \text{Gamma}(\alpha=13,\,\beta=13)
    \end{equation*}

    We calculate the posterior distribution for each hypothesis:

    \begin{multicols}{2}
        \begin{align*}
            p(H_0 \mid y) &= \int_0^{0.5} \pi(\lambda \mid y) \;d\lambda \approxeq 0.016 \\
            p(H_1 \mid y) &= \int_{0.5}^{1.5} \pi(\lambda \mid y) \;d\lambda \approxeq 0.935 \\
            p(H_2 \mid y) &= \int_{1.5}^\infty \pi(\lambda \mid y) \;d\lambda \approxeq 0.049
        \end{align*}
        \vfill\null
        \columnbreak
        \null
        \begin{tikzpicture}[
            aoc/.style={draw=none,fill,opacity=0.3},
            ]
			\begin{axis}[
					samples at={0,0.01,...,1},
					width=0.50\textwidth,
					height=0.33\textwidth,
					ytick=\empty,
					xmin=0,
					xmax=2.5,
					xlabel=$\lambda$,
					legend pos=north east,
				]

				\addplot [aoc,fill=blue,forget plot] gnuplot [raw gnuplot] { \GnuplotDefs plot [x=0:0.5] gammaPdf(x,13,13); } \closedcycle;
				\addplot [aoc,fill=green,forget plot] gnuplot [raw gnuplot] { \GnuplotDefs plot [x=0.5:1.5] gammaPdf(x,13,13); } \closedcycle;
				\addplot [aoc,fill=red,forget plot] gnuplot [raw gnuplot] { \GnuplotDefs plot [x=1.5:2.5] gammaPdf(x,13,13); } \closedcycle;
				\addplot [mark=none,black,line width=1.25pt] gnuplot [raw gnuplot] {
				\GnuplotDefs
				plot [x=0:2.5] gammaPdf(x,13,13);
				}; \addlegendentry{$\pi(\lambda \mid y)$}
            \addlegendimage{area legend,aoc,fill=blue}; \addlegendentry{$H_0$}
            \addlegendimage{area legend,aoc,fill=green}; \addlegendentry{$H_1$}
            \addlegendimage{area legend,aoc,fill=red}; \addlegendentry{$H_2$}
			\end{axis}
             % \draw[decorate,decoration={brace,raise=1.9em}] (R) -- (L) node[midway,below=2.2em] {$CI_p$};
		\end{tikzpicture}
\end{multicols}
We choose the hypothesis with the highest posterior probability: $H_1$.
\end{example}
