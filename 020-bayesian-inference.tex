%! TEX root = **/000-main.tex

\section{Bayesian Inference}

\subsection{Posterior distribution as an estimator}

The posterior distribution is a compromise between the prior
distribution (the information before observing the data) and the
likelihood function (the data information).

The posterior distribution $\pi(\theta \mid y)$ is the natural
Bayesian estimator for $\theta$ given the data $y$.

\subsection{Point Estimation}
Any measure of the location of $\pi(\theta \mid y)$ will serve as a point
estimate:
\begin{align*}
    \hat{\theta}_{pe} &= E(\theta \mid y) = \int \theta \pi(\theta \mid y) \partial\theta \\
    \hat{\theta}_{pme} &\text{ is such that } \int_{-\infty}^{\hat{\theta}_{pme}} \pi(\theta \mid y) \partial\theta = \frac{1}{2} \\
    \hat{\theta}_{pmo} &= \argmax_{\theta} \pi(\theta \mid y) \\
\end{align*}

This can also be computed numerically by simulation. Let $\theta^{(1)}, \dots , \theta^{(M)}$ be
simulations of $\theta$ from $\pi(\theta \mid y)$, then:
\begin{equation}
    \hat{\theta}_{pe} = \frac{1}{M} \sum_{i=1}^M \theta^{(i)}
\end{equation}

\paragraph{Observation} We can also define a prior point estimator using the prior distribution
instead of the posterior distribution.

\subsection{Interval Estimation}
A posterior credibility (or probability) interval of $p$ for $\theta$, $IC_p$ is
any region of $\Omega$ such that:
\begin{equation}
    p(\theta \in IC_p \mid y) = \int_{IC_p} \pi(\theta \mid y) \partial\theta = p
\end{equation}
Usually we use intervals based on percentiles.
