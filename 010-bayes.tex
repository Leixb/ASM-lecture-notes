
\section{Bayesian Model}

\begin{theorem}{Bayes' Theorem}{bayes}
	\begin{align}
		P(A \mid B) & = \frac{P(B \mid A)P(A)}{P(B)}                             \\
		P(A \mid B) & = \frac{P(B \mid A)P(A)}{\sum_{i=1}^n P(B \mid A_i)P(A_i)}
	\end{align}
\end{theorem}

\begin{definition}{Statistical Model}{}
	A \iemph{Statistical Model} $M$ is a list of Probability models $P$ indexed by
	a parameter $\theta$ (which can be a scalar, vector, matrix, etc.) that
	we know belongs to a \iemph{parameter space} $\Omega$.
	\\[1em]
	Formally:
	\begin{align}
		M = \{ P(\tilde{y}\mid\theta), \; \theta \in \Omega \}
	\end{align}
\end{definition}

Given a Statistical Model and a Data set $\tilde{y}$, we want to find
which probability model $P(\tilde{y}\mid\theta^*)$ is the most likely to
have generated the data $\tilde{y}$. Where $\theta^* \in \Omega$ is the true
value of the parameter.

\subsection{The four problems of Statistics}
\subsubsection{Data Collection}
Design how to collect the data.

This leads to choose the statistical model from which you will observe the data.
The statistical model will be determined by the nature of the data
(discrete or continuous) and by the type of sampling.

\subsubsection{Model Validation}

After choosing the model and collecting the data, we need to decide if
the model is valid or not.

That is, we have to decide if the probability model that generated the data
belongs to $M$ and therefore the model is correct.

\subsubsection{Statistical Inference}

After collecting the data and checking the model, Statistical Inference
tries to guess the true value of the parameter $\theta^*$. That is, it tries
to guess the probability model that generated the data.

It mainly consists of:
\begin{enumerate}
	\item Point estimation: Guessing the value of the parameter $\theta^*$.
	\item Interval estimation: Guessing the interval of values of the parameter $\theta^*$.
	\item Hypothesis testing: Guessing if the parameter $\theta^*$ belongs to a given set.
	\item Prediction: Guessing the value of a new data point.
\end{enumerate}

\subsubsection{Results Presentation}

Finally, the results must be interpreted and displayed in an understandable way
according to the target audience.

\subsection{Bayesian Model}

\begin{definition}{Bayesian model}{}{}
	A \iemph{Bayesian Model} is a Statistical Model where $\theta$ is treated as a random variable
	where the probability of $\theta$ is given by a prior distribution $\pi(\theta)$,
	before looking at the data (it captures prior knowledge about $\theta$).

	We call $\pi(\theta)$ the \iemph{prior distribution}.

	A \emph{Bayesian Model} is a list of probability distributions sorted from most
	likely to least likely according to the prior distribution $\pi(\theta)$:
	\begin{equation}
		M_B = \{ p(\tilde{y}\mid\theta), \; \theta \sim \pi(\theta), \; \theta \in \Omega \}
	\end{equation}
\end{definition}

\begin{definition}{Likelihood function}{}{}
	All the information that the data $y$ has about the parameter $\theta$ is in the
	\iemph{likelihood function} $L_y(\theta)$.

	The likelihood function is a function proportional to the probability distribution
	of $\tilde{y},\; p(\tilde{y}\mid\theta)$, evaluated at the data $p(y \mid \theta)$
	and expressed as a function of $\theta$:
	\begin{equation}
		L_y(\theta) \propto p(y \mid \theta)
	\end{equation}
\end{definition}

The likelihood function sorts the parameter space $\Omega$ from most likely to least,
so if $L_y(\theta_1) > L_y(\theta_2)$ it means that according to the data $y$,
the parameter $\theta_1$ is more likely than $\theta_2$.

\begin{example}{Probability of head after 10 coin tosses}{}
	\begin{alignat*}{1}
		\text{Statistical model:}   & \; \tilde y \mid \theta \sim \text{Binomial}(10, \theta), \; \theta \in [0,\,1] \\
		\text{where:}               & \; \theta \text{ is the probability of head}                                    \\
		\span \text{after 10 coin tosses, we observed 6 heads:} \; y = 6                                              \\
		\text{Likelihood function:} & \; L_y(\theta) \propto p(y \mid \theta) = \binom{10}{6} \theta^6 (1-\theta)^4
		\propto \theta^6 (1-\theta)^4
	\end{alignat*}
	%
	\begin{nscenter}
		\begin{tikzpicture}
			\begin{axis}[
					samples at={0,0.01,...,1},
					width=0.8\textwidth,
					height=0.3\textwidth,
					ytick=\empty,
					xmin=0,
					xmax=1,
					xlabel=$\theta$,
					legend pos=north west,
                    ymin=0,
				]
				\draw[red,dashed,line width=2pt] (0.6,1) -- (0.6,-0.2);
				\addplot[mark=none,line width=2pt] {binom(6,10,x)}; \addlegendentry{$L_y(\theta)$}
				\addlegendimage{draw=red,dashed,line width=2pt}; \addlegendentry{$\theta^* = 0.6$}
			\end{axis}
		\end{tikzpicture}
	\end{nscenter}
	The maximum of the likelihood function is at $\theta = 0.6$. So according to the data,
	$\theta^*$ is more likely to be 0.6 than any other value.
\end{example}

\begin{definition}{Standarized Likelihood function}{}{}
	It's the likelihood function divided by its integral over the parameter space $\Omega$,
	so that it integrates to 1 and it's a probability distribution:
	\begin{equation}
		\tilde{L}_y^{std}(\theta) = \frac{L_y(\theta)}{\int_{\Omega} L_y(\theta) \;d\theta}
	\end{equation}
\end{definition}


\begin{recap}{Prior and Likelihood}{}{}
	A Statistical model $M$ is a (not sorted) list of probability models, one for
	each value of the parameter $\theta \in \Omega$:
	\begin{equation*}
		M = \left\{ p(\tilde{y}\mid\theta), \; \theta \in \Omega \right\}
	\end{equation*}

	\begin{itemize}
		\item The \emph{prior distribution} $\pi(\theta)$ sorts the probability models of $M$
		      using \emph{prior information} about the parameter $\theta$.
		\item The \emph{likelihood function} $L_y(\theta)$ sorts the probability models of $M$
		      using the \emph{data information} $y$.
	\end{itemize}
\end{recap}

\subsection{Posterior Distribution}

\begin{definition}{Posterior Distribution}{}{}
    The \iemph{posterior distribution} $\pi(\theta \mid y)$ shows the information we have
	about the parameter $\theta$ after looking at the data $y$.

	It is calculated using Bayes' Theorem \ref{th:bayes}:
	\begin{equation}
		\pi(\theta \mid y) = \frac{p(y,\,\theta)}{p(y)} = \frac{p(y \mid \theta) \pi(\theta)}{p(y)}
	\end{equation}

	Where $p(y)$ is the \iemph{marginal distribution} of $\tilde{y}$ evaluated at the
	data $y$, which we will also call the \iemph{prior predictive distribution} evaluated at
	the data:
	\begin{equation}
		p(\tilde y = y) = p(y) = \int_{\Omega} p(y,\,\theta) \;d\theta = \int_{\Omega} p(y \mid \theta) \pi(\theta) \;d\theta
	\end{equation}

	\begin{note}
		$p(y)$ is a constant, also called the \iemph{normalization constant}, which
		makes the posterior distribution integrate to 1.
	\end{note}

	From this it follows that the posterior distribution is proportional to the
	\emph{product} of the \emph{prior distribution} $\pi(\theta)$ and the
	\emph{likelihood function} $L_y(\theta)$:
	\begin{equation}
		\pi(\theta \mid y) \propto \pi(\theta) L_y(\theta)
	\end{equation}
\end{definition}

\begin{note}
	If $\pi(\theta)$ is a uniform distribution, the posterior distribution is the
	standardized likelihood function $L_y^{std}(\theta)$.
\end{note}

\begin{example}{Probability of head after 10 coin tosses $y=7$}{}
	We assume that the probability of head is close to 50\%, so we chose an
	informative prior distribution:
	\begin{alignat*}{1}
		\text{Statistical model:}      & \; \tilde y \mid \theta \sim \text{Binomial}(10, \theta), \; \theta \in [0,\,1] \\
		\text{Prior distribution:}     & \; \theta \sim \pi(\theta) = \text{Beta}(50,\,50)                               \\
		% \text{where:}             & \; \theta \text{ is the probability of head}                                    \\
		\span \text{after 10 coin tosses, we observed 7 heads:} \; y = 7                                                 \\
		% \span \text{From $y$ we calculate likelihood and posterior:} \\
		\text{Likelihood function:}    & \; L_y(\theta) \propto p(y \mid \theta) = \binom{10}{7} \theta^7 (1-\theta)^3
		\propto \theta^7 (1-\theta)^3                                                                                    \\
		\text{Posterior distribution:} & \; \pi(\theta \mid y) = \text{Beta}(57,\,53)
	\end{alignat*}
	%
	\begin{nscenter}
		\begin{tikzpicture}
			\begin{axis}[
					samples at={0,0.01,...,1},
					width=0.8\textwidth,
					height=0.33\textwidth,
					ytick=\empty,
					xmin=0,
					xmax=1,
					xlabel=$\theta$,
					legend pos=north west,
                    ymin=0,
                    ymax=9,
                    legend cell align={left},
				]
				\addplot [mark=none,dashed,blue,line width=2pt]gnuplot [raw gnuplot] {
				\GnuplotDefs
				plot [x=0:1] beta(x,50,50);
				}; \addlegendentry{$\pi(\theta)$}
				\addplot[mark=none,blue,line width=2pt] gnuplot [raw gnuplot] {
						\GnuplotDefs
						plot [x=0:1] beta(x,57,53);
					}; \addlegendentry{$\pi(\theta \mid y)$}
				\addplot[mark=none,dotted,red,line width=2pt] {binom(7,10,x)*10}; \addlegendentry{$L_y(\theta)$}
			\end{axis}
		\end{tikzpicture}
	\end{nscenter}
\end{example}

\begin{example}{Probability of head after 10 coin tosses $y=7$ v2}{}
	We take a less-informative prior distribution, since we are not sure that the coin
	is fair:
	\begin{alignat*}{1}
		\text{Statistical model:}      & \; \tilde y \mid \theta \sim \text{Binomial}(10, \theta), \; \theta \in [0,\,1] \\
		\text{Prior distribution:}     & \; \theta \sim \pi(\theta) = \text{Beta}(2,\,2)                                 \\
		% \text{where:}             & \; \theta \text{ is the probability of head}                                    \\
		\span \text{after 10 coin tosses, we observed 7 heads:} \; y = 7                                                 \\
		% \span \text{From $y$ we calculate likelihood and posterior:} \\
		\text{Likelihood function:}    & \; L_y(\theta) \propto p(y \mid \theta) = \binom{10}{7} \theta^7 (1-\theta)^3
		\propto \theta^7 (1-\theta)^3                                                                                    \\
		\text{Posterior distribution:} & \; \pi(\theta \mid y) = \text{Beta}(9,\,5)
	\end{alignat*}
	\vspace{-1em}
	\begin{nscenter}
		\begin{tikzpicture}
			\begin{axis}[
					samples at={0,0.01,...,1},
					width=0.8\textwidth,
					height=0.33\textwidth,
					ytick=\empty,
					xmin=0,
					xmax=1,
					xlabel=$\theta$,
					legend pos=north west,
                    ymin=0,
                    ymax=9,
                    legend cell align={left},
				]
				\addplot [mark=none,dashed,blue,line width=2pt]gnuplot [raw gnuplot] {
				\GnuplotDefs
				plot [x=0:1] beta(x,2,2);
				}; \addlegendentry{$\pi(\theta)$}
				\addplot[mark=none,blue,line width=2pt] gnuplot [raw gnuplot] {
						\GnuplotDefs
						plot [x=0:1] beta(x,9,5);
					}; \addlegendentry{$\pi(\theta \mid y)$}
				\addplot[mark=none,dotted,red,line width=2pt] {binom(7,10,x)*10}; \addlegendentry{$L_y(\theta)$}
			\end{axis}
		\end{tikzpicture}
	\end{nscenter}
	\begin{note}
		With a less informative prior, the posterior distribution is more affected by the
		likelihood function. If we were to use a flat prior, the posterior would be equal
		to the likelihood function.
	\end{note}
\end{example}

\begin{recap}{Posterior distribution}{}
	\centering
	\begin{tikzpicture}[every node/.style={rectangle,draw,fill=white}]
		\node[align=center] (model) {%
			Bayesian model\\[1em]
			$\begin{aligned}
					\tilde y \mid \theta & \sim p(\tilde y \mid \theta) \\
					\theta               & \sim \pi(\theta)
				\end{aligned}$};

		\node[below=of model] (bayes) {Bayes' theorem};
		\node (data) [left=of bayes] {Data $y$};

		\node[below=of bayes,align=center] (updated) {%
			Bayesian model (updated)\\[1em]
			$\begin{aligned}
					\tilde y \mid \theta & \sim p(\tilde y \mid \theta) \\
					\theta \mid y        & \sim \pi(\theta \mid y)
				\end{aligned}$};

		\node[right=of updated,align=center,draw=none,fill=none] (post) {Posterior distribution \\[1em]
			$ \pi(\theta \mid y) = \frac{L_y(\theta) \pi(\theta)}{\int L_y(\theta)\pi(\theta)\;d\theta} $};

		\draw[->,line width=5pt] (model) -- (bayes) -- (updated);
		\draw[->,line width=5pt] (data) -- (bayes);
		\draw[->,line width=3pt,dashed] (updated.345) -- (post);
	\end{tikzpicture}
\end{recap}

\subsection{Prior and posterior predictive distributions}%

\begin{definition}{Prior predictive distribution}{}
	The \iemph{marginal distribution} of $\tilde{y}$, called the \iemph{prior predictive distribution},
	represents our knowledge and uncertainty about the sample space $\tilde{y}$ before looking at the data $y$:
	\begin{equation}
		p(\tilde y) = \int_{\Omega} p(\tilde y ,\, \theta) \;d\theta = \int_{\Omega} p(\tilde y \mid \theta) \pi(\theta) \;d\theta
	\end{equation}

	\paragraph{Observations}
	\begin{itemize}
		\item $p(\tilde y) = E_{\pi(\theta)}[p(\tilde y \mid \theta)]$ is a weighted average of the
		      probability models of $M$ where the weights are determined by the prior distribution $\pi(\theta)$.
		\item $p(\tilde y)$ is a probability distribution that allows us to make predictions before
		      observing the data.
	\end{itemize}
\end{definition}

If we know something about the parameters, it means that we have some prior information about the data
and vice versa. The prior predictive distribution translates the information about the
parameter space into the sample space.

The prior's parameters can be chosen in two ways:
\begin{itemize}
	\item Prior distribution $\pi(\theta)$ if the information we have is about the parameter space.
	\item Prior predictive distribution $p(\tilde y)$ if the information we have is about the sample space.
\end{itemize}

\begin{definition}{Posterior predictive distribution}{post-pred}
	Once the data $y$ is observed, we update the model to reflect the new information,
	changing the prior distribution $\pi(\theta)$ into the posterior distribution $\pi(\theta \mid y)$.
	Which means that the prior predictive distribution $p(\tilde y)$ is replaced by the
	\iemph{posterior predictive distribution} $p(\tilde y \mid y)$:
	\begin{equation}
		p(\tilde y \mid y) = \int_{\Omega} p(\tilde y \mid \theta) \pi(\theta \mid y) \;d\theta
	\end{equation}

	\paragraph{Observations}
	\begin{itemize}
		\item $\pi(\theta y \mid y)$ shows everything we know about the parameter.
		\item $p(\tilde y \mid y)$ shows everything we know about the data behavior.
	\end{itemize}
\end{definition}

Predictive distributions can be calculated in two ways:
\begin{itemize}
	\item Analytically: solving integrals:
	      \begin{align*}
		      p(\tilde y)        & = \int_{\Omega} p(\tilde y \mid \theta) \pi(\theta) \;d\theta        \\
		      p(\tilde y \mid y) & = \int_{\Omega} p(\tilde y \mid \theta) \pi(\theta \mid y) \;d\theta
	      \end{align*}
	\item Numerically (by simulation): sampling from the posterior distribution:
	      \begin{enumerate}
		      \item Decide on a number of simulations $M$ (the larger the better).
		      \item Sample $\theta^{(j)}$ from $\pi(\theta \mid y)$.
		      \item Sample $\tilde y^{(j)}$ from $p(\tilde y \mid \theta^{(j)})$.
		      \item Repeat $j=1,\ldots,M$.
	      \end{enumerate}
	      These values $\tilde y^{(j)}$ can be used to calculate everything we want to know:
	      mean, variance, quantiles, shape of the distribution, etc.
\end{itemize}

\begin{recap}{Predictive distributions}{}
	\centering
	\begin{tikzpicture}[every node/.style={rectangle,draw,fill=white}]
		\node[align=center] (model) {%
			Bayesian model\\[1em]
			$\begin{aligned}
					\tilde y \mid \theta & \sim p(\tilde y \mid \theta) \\
					\theta               & \sim \pi(\theta)
				\end{aligned}$};

		\node[below=of model] (bayes) {Bayes' theorem};
		\node (data) [left=of bayes] {Data $y$};

		\node[below=of bayes,align=center] (updated) {%
			Bayesian model (updated)\\[1em]
			$\begin{aligned}
					\tilde y \mid \theta & \sim p(\tilde y \mid \theta) \\
					\theta \mid y        & \sim \pi(\theta \mid y)
				\end{aligned}$};

		\node[right=of model,draw=none,fill=none] (prior) {
			$p(\tilde y)$ (prior predictive dist.)
		};
		\node[right=of updated,draw=none,fill=none] (post) {
			$p(\tilde y \mid y)$ (posterior predictive dist.)
		};

		\draw[->,line width=5pt] (model) -- (bayes) -- (updated);
		\draw[->,line width=5pt] (data) -- (bayes);
		\draw[->,line width=3pt,dashed] (model) -- (prior);
		\draw[->,line width=3pt,dashed] (updated) -- (post);
	\end{tikzpicture}
\end{recap}

\subsection{Choosing the prior}

The prior distribution $\pi(\theta)$ is a subjective choice. It should be based on previous
studies and subjective knowledge of experts. In case of not having information or
not being able to agree on the knowledge of experts, we will use a \iemph{non-informative prior}.

\begin{definition}{Conjugate prior}{}
	A prior distribution $\pi(\theta)$ is the \iemph{conjugate} prior of a statistical model
	if the posterior distribution $\pi(\theta \mid y)$ is of the same family as the prior.

	\begin{table}[H]
		\caption{Examples of conjugate priors}
		\colorbox{white}{%
			\begin{tabular}{ccc}
				\toprule
				Model                & Prior                 & Posterior                        \\
				\midrule
				Binomial($n,\theta$) & Beta($\alpha,\beta$)  & Beta($\alpha + y,\beta + n - y$) \\
                Poisson($\lambda$)   & Gamma($\alpha,\beta$) & Gamma($\alpha + n,\beta + \sum_{i=1}^n y_i$)    \\
				\bottomrule
			\end{tabular}%
		}
	\end{table}

\end{definition}

\paragraph{Informative priors}
To chose informative priors, we can either do trial and error, drawing
$\pi(\theta)$ or solving a system of equations based on moments and or quantiles.

\paragraph{Non-informative priors}
The most common are a flat prior (Laplace's prior) and a conjugate prior with
\emph{huge} variance (e.g. Gamma($\alpha=0.001,\beta=0.001$)).
