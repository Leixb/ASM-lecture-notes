\chapter{Time Series}
\section{Introduction}

\begin{definition}{Time series}{time-series}\index{time-series}
	Ordered sequence of observations of the same phenomenon. Typically
	measured at equally spaced successive instants of time.
	\begin{equation*}
		\{X_t\}_{t=1,\ldots,T} = \{X_1, X_2, \ldots, X_T\}
	\end{equation*}
\end{definition}

\subsection{Motivation}

Describing and forecasting time series is crucial in different areas of
knowledge; including finance, econometrics, signal processing and a long etc.

\subsection{Objectives}

\begin{itemize}
	\item \textbf{Description}: Describe temporal patterns in a time series: regular and/or
	      seasonal effects, cyclicity, trends, outliers, sudden changes, breaks, \dots
	\item \textbf{Estimation}: Estimate the values of the time series parameters.
	\item \textbf{Validation}: Validate the estimated parameters and decide if the estimated
	      parameters are significant or not.
	\item \textbf{Prediction/Forecasting}: Predict future values of the time series.
\end{itemize}

\begin{example}{AirPassengers}{AirPassengers}
	Monthly totals of international airline passengers in USA, 1949
	to 1960 (Box \& Jenkins, 1976).
	\begin{verbatim}
##      Jan Feb Mar Apr May Jun Jul Aug Sep Oct Nov Dec
## 1949 112 118 132 129 121 135 148 148 136 119 104 118
## 1950 115 126 141 135 125 149 170 170 158 133 114 140
## 1951 145 150 178 163 172 178 199 199 184 162 146 166
## 1952 171 180 193 181 183 218 230 242 209 191 172 194
## 1953 196 196 236 235 229 243 264 272 237 211 180 201
## 1954 204 188 235 227 234 264 302 293 259 229 203 229
## 1955 242 233 267 269 270 315 364 347 312 274 237 278
## 1956 284 277 317 313 318 374 413 405 355 306 271 306
## 1957 315 301 356 348 355 422 465 467 404 347 305 336
## 1958 340 318 362 348 363 435 491 505 404 359 310 337
## 1959 360 342 406 396 420 472 548 559 463 407 362 405
## 1960 417 391 419 461 472 535 622 606 508 461 390 432
\end{verbatim}
	\begin{nscenter}
		\begin{tikzpicture}
			\begin{axis}[
					width=0.9\textwidth,
					height=0.4\textwidth,
					date coordinates in=x,
					table/col sep=comma,
					xlabel=Time,
					ylabel=Passengers,
					xticklabel=\year,
					xtick distance=366*2,
					mark size=1pt,
					mark=o,
					xmajorgrids=true,
					minor x tick num=1,
					xminorgrids=true,
				]
				\addplot table[x=Month,y=Passengers] {data/AirPassengers.csv};
			\end{axis}
		\end{tikzpicture}
	\end{nscenter}
\end{example}

\subsection{Exploratory Data analysis}

Plot of the series and identification of the components:

\begin{definition}{Trend ($T_t$)}{trend}\index{trend}
	Long term tendency of the series.

	Moving average of order $s$:
	\begin{equation*}
		T_t = \frac{1}{s} \sum_{i=1}^s X_{t-s/2+i}
	\end{equation*}
\end{definition}

\begin{definition}{Seasonal ($S_t$)}{seasonal}\index{seasonal}
	Pattern repeated periodically with the same period.

	\index{detrended series}
	\paragraph{Seasonal index} Mean for each period of detrended series ($X_t - T_t$).
\end{definition}

\begin{definition}{Cycle ($C_t$)}{cycle}\index{cycle}
	Pattern repeated periodically with \emph{non-constant} period.

	\vspace{1em}
	\begin{marker}
		Not easy to model due to the changing period.
	\end{marker}
\end{definition}

\begin{definition}{Random ($w_t$)}{random}\index{random}
	Random noise.

	\paragraph{Remainder} ($X_t - T_t - S_t - C_t$)
\end{definition}

Our \emph{goal} is to find a mathematical model that reflects the behavior of the observed
data.

\subsection{Modeling}

\subsubsection{Additive model}
We add the different components:\index{additive model}
\begin{equation}
	X_t = T_t + S_t + C_t + w_t \tag{additive}
\end{equation}

\begin{figure}[H]
	\begin{tikzpicture}
		\begin{axis}[
				width=0.9\textwidth,
				height=0.5\textwidth,
				date coordinates in=x,
				xticklabel=\year,
				xtick distance=366*2,
				xlabel=Time,
				ylabel=Passengers,
				ymajorgrids=true,
				yminorgrids=true,
				xmajorgrids=true,
				minor x tick num=1,
				xminorgrids=true,
				legend pos=north west,
			]
			\directlua{dofile("lua/decomposition.lua")}

			\addlegendentry{$X$ (data)}
			\addlegendentry{$T$ (trend)}
			\addlegendentry{$S$ (seasonal)}
			\addlegendentry{$T+S$}
			\addlegendentry{$w = X-T-S$ (random)}
		\end{axis}
	\end{tikzpicture}
	\caption{Decomposition of the \texttt{AirPassengers} series from example~\ref{example:AirPassengers}}
\end{figure}

\subsubsection{Deterministic model}
The expected value of $X_t$ depends on a parametric function $F$ of $t$ and the random
component does not depend on the previous values.
\begin{equation}
	X_t = F(t) + Z_t \qquad Z_t \sim N(0,\, \sigma_z^2) \tag{deterministic}
\end{equation}

\subsubsection{Stochastic model}
The expected value of $X_t$ depends on the previous values $X_{t-1}, X_{t-2}, \dots$ and/or
the previous random components $Z_{t-1}, Z_{t-2}, \dots$ plus a random component independent
of the past $Z_t$.
\begin{equation}
	X_t = G(X_{t-1}, X_{t-2}, \dots, Z_{t-1}, Z_{t-2}, \dots) + Z_t \qquad Z_t \sim N(0,\, \sigma_z^2) \tag{stochastic}
\end{equation}

\subsection{Box-Jenkins methodology}

\begin{nscenter}
	\begin{tikzpicture}[
			box/.style={draw, rectangle, minimum width=5cm, minimum height=1.5cm, align=center},
			decision/.style={diamond, minimum width=2cm, minimum height=1cm, align=center},
			box_l/.style={align=left},
		]
		\node[box, rounded corners] (ti) {Tentative\\Identification};
		\node[box, below=of ti] (est) {Estimation};
		\node[box, below=of est] (val) {Diagnostic\\ Checking};
		\node[box, decision, below=of val] (dec) {Model\\ok?};
		\node[box, rounded corners, below=of dec] (fin) {Final Model};

		\node[box_l, right=of ti] {Time Series Plot \\Range-Mean Plot\\ACF and PACF};
		\node[box_l, right=of est] {Least Squares or\\Maximum Likelihood};
		\node[box_l, right=of val] {Residual Analysis\\and Forecasts};
		\node[box_l, right=of fin] {Forecasting\\Explanation};


		\draw[->] (ti) -- (est);
		\draw[->] (est) -- (val);
		\draw[->] (val) -- (dec);
		\draw[->] (dec) edge[edge label={Yes}] (fin);

        \coordinate (fb) at ($(val.west)+(-3em,0)$);
        \node (no) at ($(dec.west)+(-2em,0)$) {No};
        \draw[->] (dec.west) -- (no) -| (fb) |- (ti.west);
	\end{tikzpicture}
\end{nscenter}

\subsection{Distribution of a general stochastic process}

First and second moments for the multivariate distribution of $\{X_t\}_{t=1\dots T}$:
\begin{align*}
	\mathbb{E}[(X_1, \dots, X_T)] & = (\mu_1, \dots, \mu_T)                                              \\
	Var[(X_1, \dots, X_T)]        & = \begin{pmatrix}
		                                  \sigma_1^2   & \sigma_{1,2} & \sigma_{1,3} & \dots  & \sigma_{T,1} \\
		                                  \sigma_{2,1} & \sigma_2^2   & \sigma_{2,3} & \dots  & \sigma_{T,2} \\
		                                  \sigma_{3,1} & \sigma_{3,2} & \sigma_3^2   & \dots  & \sigma_{T,3} \\
		                                  \vdots       & \vdots       & \vdots       & \ddots & \vdots       \\
		                                  \sigma_{T,1} & \sigma_{T,2} & \sigma_{T,3} & \dots  & \sigma_T^2
	                                  \end{pmatrix}
\end{align*}

Parameters of the model:
\begin{itemize}
    \item $T$ values for the mean $\mathbb{E}(X_t) = \mu_t$
    \item $T$ values for the variances $Var(X_t) = \sigma_t^2$
    \item $T(T-1)$ values for the covariances $Cov(X_t, X_s) = \sigma_{t,s}$
\end{itemize}

\subsection{Stationary Series}

Strict Stationary Process or Series has the following properites:
The joint distribution of the whole series does not depend on the time
origin:
\begin{equation}
    F_{X_1, \dots, X_t}(x_1, \dots, x_t) = F_{X_{1+s}, \dots, X_{t+s}}(x_{1+s}, \dots x_{t+s}) \forall t,s
\end{equation}

\subsection{Weakly Stationary process}
The two first moments of the multivariate distribution of the whole series
does not depend on the time origin:
\begin{itemize}
    \item Constant mean $\mu$.
    \item Constant variance $\sigma^2$.
    \item Constant auto-covariance structure $\sigma_{t,s} = \sigma_{\lVert t - s \rVert}$.
    \item The latter refers to the covariance between $X_t$ and $X_{t-1}$ being equal to 
        the covariance between $X_{t-s}$ and $X_{t-s-1}$.
\end{itemize}

Weakly Stationary process + Gaussian multivariate distribution $\Longrightarrow$ Strict Stationary process.

\begin{question}{Is our data stationary? How can we detect?}{}
    \begin{itemize}
        \item Plot the data.
        \item Identify no stationary components (trends, seasonal patterns, cycles).
        \item Transform the series to remove those components.
        \item For the transformed (stationary) series, plot and analyze the sample autocorrelation.
    \end{itemize}
\end{question}

\begin{marker}
    It is very common that the variance of the series increases when the level of the series rises.
\end{marker}

\begin{figure}[H]
    \begin{tikzpicture}
        \begin{axis}[
                width=0.48\textwidth,
                height=0.4\textwidth,
                date coordinates in=x,
                table/col sep=comma,
                xlabel=Time,
                ylabel=Passengers,
                xticklabel=\year,
                xtick distance=366*4,
                mark size=1pt,
                mark=o,
                xmajorgrids=true,
                minor x tick num=1,
                xminorgrids=true,
                title={\bfseries Non-Constant Variance},
            ]
            \addplot table[x=Month,y=Passengers] {data/AirPassengers.csv};
        \end{axis}
    \end{tikzpicture}
    \begin{tikzpicture}
        \begin{axis}[
                width=0.48\textwidth,
                height=0.4\textwidth,
                table/col sep=comma,
                xlabel=Time,
                ylabel=$\text{CO}_2$ (ppm),
                mark size=1pt,
                mark=o,
                xmajorgrids=true,
                minor x tick num=1,
                xminorgrids=true,
                enlargelimits=false,
                x tick label style={/pgf/number format/.cd, set thousands separator={}},
                title={\bfseries Constant Variance},
            ]
            \addplot+[red] table[mark=none,x=decimal date,y=average] {./data/co2_mm_mlo.csv};
        \end{axis}
    \end{tikzpicture}
    \caption{Examples of constant and non-constant variance}
\end{figure}

\subsection{Tools to diagnose the non-constant variance}

\paragraph{Mean-Variance Plot} Calculate the mean and variance of consecutive
groups of 8-12 observations. Plot the variance against the mean of each group.

\begin{figure}[H]
	\begin{tikzpicture}
		\begin{axis}[
				width=0.9\textwidth,
				height=0.5\textwidth,
				xlabel=Mean,
				ylabel=Variance,
				% ymajorgrids=true,
				% yminorgrids=true,
				% xmajorgrids=true,
			]
			\directlua{dofile("lua/variance-mean.lua")}
		\end{axis}
	\end{tikzpicture}
	\caption{Mean-Variance plot of the \texttt{AirPassengers} series}
\end{figure}

\paragraph{Boxplot for periods} Represent the boxplot for each group of 8-12
observations. The height of the box (IQR) is a robust estimate of variability.

\begin{figure}[H]
	\begin{tikzpicture}
		\begin{axis}[
				width=0.9\textwidth,
				height=0.5\textwidth,
				xlabel=Time,
				ylabel=Passengers,
				% ymajorgrids=true,
				% yminorgrids=true,
				% xmajorgrids=true,
                boxplot/draw direction=y,
                xtick={2,4,6,8,10,12},
                xticklabels={1950,1952,1954,1956,1958,1960},
                % boxplot/median/.style={line width=2pt},
			]
			\directlua{dofile("lua/boxplot.lua")}
		\end{axis}
	\end{tikzpicture}
	\caption{Boxplot for periods of the \texttt{AirPassengers} series}
\end{figure}

\begin{itemize}
    \item If the variance is similar for all the groups $\Longrightarrow$ No scale transformation.
    \item If the variance is higher for higher values of the mean $\Longrightarrow$ Change the scale.
\end{itemize}

\subparagraph{Box-Cox transformation}
\begin{equation}
    Y_t = \begin{cases}
        \frac{X_t^\lambda - 1}{\lambda} & \text{if } \lambda \neq 0 \\
        \ln(X_t) & \text{if } \lambda = 0
    \end{cases} \qquad \lambda \in [-1,\,2]
\end{equation}
\begin{marker}
    Usually, the logarithmic transformation is used since it easier to interpret.
\end{marker}

